\documentclass{article}
\usepackage[utf8]{inputenc}
\usepackage{lmodern}
\usepackage[T1]{fontenc}
\usepackage[english]{babel}
\usepackage{amsmath}
\usepackage{amsthm}
\usepackage{amssymb}

\usepackage[backend=biber,style=numeric]{biblatex}
\usepackage{csquotes}
\addbibresource{viitteet.bib}

\title{The order of Euler's totient function}
\author{Elli Kiiski}
\date{2021}

\theoremstyle{definition}
\newtheorem{definition}[subsection]{Definition}
\newtheorem{notation}[subsection]{Notation}
\newtheorem{agreement}[subsection]{Agreement}
\newtheorem{theorem}[subsection]{Theorem}

\begin{document}

\maketitle

\newpage

\section{Summary}

Viittaus \cite{HardyWright} toinenkin \cite{Saksman}

\section{Introduction}

\section{Notation and definitions}

\begin{agreement}{\emph{About numbers}}

Unless defined otherwise, all introduced variables $a, b, c, ...$ are integers. The set of natural numbers $\mathbb{N}$ consists of positive integers, meaning $0 \not\in \mathbb{N}$.

\end{agreement}

\begin{notation}{\emph{Divisibility}}

Let $a$ and $b$ be such that $b$ is divisible by $a$. This is denoted by $a \vert b$.

\end{notation}

\begin{definition}{\emph{Greatest common divisor}}

Let $a \in \mathbb{N}$ and $b \in \mathbb{N}$. There is a unique $d \in \mathbb{N}$ with following properties:

\begin{enumerate}
 \item $d \vert a$ and $d \vert b$
 \item if $c \vert a$ and $c \vert b$, then $c \vert d$
\end{enumerate}

The number $d$ is called the greatest common divisor of $a$ and $b$, denoted by $gcd(a,b) = d$.

\end{definition}

\begin{definition}{\emph{Prime number}}

Integer $p\in\mathbb{N}$ is a prime, if $p \geq 2$ and for every $k\in\mathbb{N}$ holds that if $k \vert p$ then $k\in{1, p}$. The set of prime numbers is denoted by $\mathbb{P}$.

In other words, all integers greater than than $1$, which are only divisible by themself and $1$, are primes.

%Toisin sanoen alkulukuja ovat kaikki lukua $1$ suuremmat luonnolliset luvut, jotka ovat jaollisia vain itsellään ja luvulla $1$.

\end{definition}

\begin{definition}{\emph{Co-prime}}

If $gcd(a,b) = 1$, $a$ and $b$ are called co-primes or relative primes.

\end{definition}

\begin{definition}{\emph{Multiplicative number theoretic function}}

Function $f: \mathbb{N} \rightarrow \mathbb{R}$ is called number theoretic function. It is multiplicative if $f(ab) = f(a)f(b)$ when $gcd(a, b)=1$.

\end{definition}

\section{Euler's totient function and its properties}

Euler's totient function is a number theoretic function. KESKEN

\begin{definition}{\emph{Euler's totient function $\phi: \mathbb{N} \rightarrow \mathbb{N}$}}

It is set that $\phi(1) = 1$. For all $n \geq 2$, $\phi(n)$ is the number of integers $a \in \{1,2,...,n\}$, for which $gcd(a,n) = 1$.

%Toisin sanoen Eulerin $\phi$-funktion arvo luonnollisella luvulla $n$ on sitä pienempien luonnolisten lukujen määrä, jotka ovat alkulukuja sen suhteen.

\end{definition}

That is, the value of the totient function at $n \in \mathbb{N}$ is the number of co-primes of $n$ smaller than it. KENTIES SELKEÄMPI SELITYS ON OLEMASSA

\newpage

\begin{theorem}{\emph{Euler's product formula}}

\begin{equation*}
    \phi(n) = n \prod_{p \vert n} (1 - \frac{1}{p})
\end{equation*}

where $\prod_{p \vert n} (1 - \frac{1}{p})$ means the product over \emph{distinct} primes that divide $n$.

\begin{proof}

KIKKI

\end{proof}

\end{theorem}

\begin{theorem}{\emph{Totient function and primes}}

For every $p \in \mathbb{P}$ holds $\phi(p) = p-1$

\begin{proof}

KIKKI

\end{proof}

\end{theorem}

\section{The limits of Euler's totient function}

As shown, there is an exact formula for the rather verbally defined totient function $\phi(n)$. Though, using it requires factorization of $n$, which seems to cause the difficulty to estimate its size as $n$ gets bigger. VÄHÄN AIHEEN VIERESTÄ JA OUTO

For example, let $n = 2^p - 1 \in \mathbb{P}$ where also $p \in \mathbb{P}$ ($n$ is so called Mersenne prime). By theorem \ref{Totient function and primes} we know $\phi(n) = n - 1$. On the other hand, from Euler's product formula follows that $\phi(n+1) = \phi(2^p) = 2^p(1-\frac{1}{2}) = \frac{2^p}{2} = \frac{n+1}{2}$. Now we see that while $n$ and $n+1$ differ from each other only insignificantly, $\phi(n+1)$ is half the value of $\phi(n)$. MERSENNE PRIME EHKÄ PITÄÄ AVATA JA EHKÄ PERUSTELLA ETTÄ SELLASIA ON OLEMASSA

As we see the size of Euler's totient function fluctuates as it grows, making its size quite difficult to define. After shortly proving the trivial upper limit of the totient function, we move on to the more complicated lower limit. PAREMPI AASINSILTA TARVITAAN, EHKÄ VOI JOPA SIIRTÄÄ SEURAAVAN ALAOTSIKON ALLE

\subsection{Upper limit of Euler's totient function}

AASINSILTA HAKUSESSA

\begin{theorem}{\emph{Upper limit of the totient function}}

$\phi(n) \leq n-1$ for every $n \geq 2$.

\begin{proof}

By definition $\phi(n) \leq n$ beacause there are $n$ elements in the set $\{1,2,...,n\}$. Also, for every $n \geq 2$ holds $gcd(n,n) = n \neq 1$. Thus, $\phi(n) \leq n-1$.

On the other hand, according to theorem \ref{Totient function and primes}, $\phi(p) = p-1$ for every $p\in\mathbb{P}$. This means that $\phi(n) = n-1$ is, in fact, the sumpremum of Euler's totient function. TARKISTA SUPREMUM MÄÄRITELMÄ. PITÄÄ VARMAAN MAINITA VIELÄ, ETTÄ ALKULUKUJA ON ÄÄRETTÖMÄSTI.

\end{proof}

\end{theorem}

\subsection{Lower limit of Euler's totient function}

The lower limit of the totient function being anything but trivial, as given a little example above, let's start examining it by considering some simple lower limit candidates.

\subsubsection{Are there such integers $n$ that $\phi(n) < \sqrt(n)$?}

Let's begin with $\sqrt{n}$. Is there such large number $n$ that $\phi(n) < \sqrt(n)$? When checking the values of $\phi(n)$ for smaller $n$, we see that at least with $n=6$ the statement is true, as $\phi(6)=2<\sqrt{6}$. After that, however, the values seem to be consistently above the corresponding squareroot value.

Reasonable guess would be to assume that $\sqrt{n}$ is a lower limit for $\phi(n)$ when $n \rightarrow \infty$. With more precise examination, we see that is indeed the case.

\section{Average of Euler's totient function}

\section{Asiaaa}

\section{References}

\printbibliography[heading=none]

\end{document}