\documentclass{article}
\usepackage{kandityyli}

\usepackage[backend=biber,style=numeric]{biblatex}
\usepackage{csquotes}
\addbibresource{viitteet.bib}

\begin{document}

\begin{titlepage}
\setlength{\parindent}{0mm}
\Large
\textsc{University of Helsinki \\
Faculty of Science\\
Department of Mathematics and Statistics}
\vspace{5mm}
\hrule height2pt

\begin{center}
\vfill
\Large Bachelor's thesis  \\
\vspace{3mm}
\huge 
\textbf{The Order of Euler's totient function}\\
\vspace{3mm}
\Large Elli Kiiski
\vfill
\end{center}

\hrule height2pt
\vspace{5mm}
Bachelor's Programme in Mathematical Sciences \\[2mm]
Supervisor: Eero Saksman
\hfill
\today
\end{titlepage}

\tableofcontents
\thispagestyle{empty}
\clearpage

\pagenumbering{arabic}
\section{Introduction}

\textcolor{gray}{Tähän\\mun pitää\\keksiä joku\\alkusepitys varmaan}

It is beneficial to go through the basic definitions concerning the essential areas before considering the totient function itself.

The variables $k$, $m$, $n$ and $p$ are usually natural numbers, $p$ denoting particularly a prime number. Here the set of natural numbers $\N$ consists of positive integers, meaning $0 \not\in \N$. Also, in chapter \ref{apujutut} the notation $\lfloor x\rfloor$ is used to denote the integer part and $\{x\}$ the decimal part of a real number $x$. In other words, $x=\lfloor x\rfloor+\{x\}$.

\begin{definition}
Divisibility

If $b=ka$ for some integer $k$, $b$ is divisible by $a$. This is denoted by $a \,\vert\, b$.

\end{definition}

\begin{theorem}
Greatest common divisor

Let $a \in \N$ and $b \in \N$. There is a unique $d \in \N$ with following properties:

\begin{enumerate}
 \item $d \,\vert\, a$ and $d \,\vert\, b$
 \item if $c \,\vert\, a$ and $c \,\vert\, b$, then $c \,\vert\, d$
\end{enumerate}

The number $d$ is called the greatest common divisor of $a$ and $b$, denoted by $gcd(a,b) = d$.

\begin{proof}
This is proved in \textit{Fundamentals of Number theory} \cite{LeVeque} as Theorem 2.1 (ch. 2.1, p. 31).
%\textcolor{blue}{LeVeque: thm 2.1, chp 2.1, p. 31}
\end{proof}
\end{theorem}

\begin{definition}
Congruence

Let $m\not=0$. If $m\,\vert\,(a-b)$, we say $a$ is congruent to $b$ modulo $m$. It is denoted by $a\equiv b \pmod{m}$.
\end{definition}

\begin{definition}
Complete residue system (mod m)

The set $a_0,a_1,\dots,a_{m-1}$ forms a complete residue system if
\begin{equation*}
    a_i \equiv i \pmod{m}
\end{equation*}f
or all $i\in\{0,1,\dots,m-1\}$.
\end{definition}

\begin{definition}
Prime number

Integer $p\in\N$ is a prime, if $p \geq 2$ and for every $k\in\N$ holds that if $k \,\vert\, p$ then $k\in\{1, p\}$. The set of prime numbers is denoted by $\Prime$.

%In other words, all integers greater than $1$, only divisible by themself and $1$, are primes.

\end{definition}

\begin{definition}
Co-prime

If $gcd(a,b) = 1$, $a$ and $b$ are called co-primes or relative primes.
\end{definition}

\begin{definition}
Multiplicative number theoretic function

Function $f\colon \N \rightarrow \R$ is called number theoretic function. It is multiplicative if $f(ab) = f(a)f(b)$ when $gcd(a, b)=1$.

\end{definition}

\section{Euler's totient function and its properties}

Now that we have revised some of the basic concepts in number theory, let us introduce Euler's totient function (also known as Euler's $\phi$-function) itself. It is a number theoretic function that counts the positive co-primes of a given number that are less or equal to it. We start by formally defining the function and then showing a few of its properties.

\begin{definition}
Euler's totient function $\phi: \N \rightarrow \N$
%\textcolor{blue}{Hardy-Wright: chapter 5.5, p. 63}

We set $\phi(1) = 1$. For all $n \geq 2$, $\phi(n)$ is the number of integers $a \in \{1,2,\dots,n\}$, for which $gcd(a,n) = 1$.

\end{definition}

%That is, the value of $\phi(n)$ is the number of positive co-primes of $n$ less or equal to $n$.

\begin{figure}[!htb]
    \centering
    \includegraphics[width =120mm]{kuva1-phifunc.eps}
    \caption{Graph of Euler's totient function when $n\in\{1,2,\dots,100\}$}.
    \label{fig:k1}
\end{figure}

\clearpage

\begin{theorem}
%\textcolor{blue}{Hardy-Wright: chapter 5.5, theorem 60, p. 64}
Euler's totient function is multiplicative, i.e.
\begin{equation*}
    gcd(m,n)=1 \quad\Rightarrow\quad \phi(mn)=\phi(m)\phi(n)\,.
\end{equation*}

\begin{proof}
Assume $m>1$, $n>1$ and $gcd(m,n)=1$. Consider the array

\begin{table}[!htb]
    \centering
    \begin{tabular}{ccccc}
        $0$ & $1$ & \dots & $m-2$ & $m-1$ \\
        $m$ & $m+1$ & \dots & $m+(m-2)$ & $m+(m-1)$\\
        \vdots & \vdots & \vdots & \vdots & \vdots\\
        $(n-2)m$ & $(n-2)m+1$ & \dots & $(n-2)m+(m-2)$ & $(n-2)m+(m-1)$\\
        $(n-1)m$ & $(n-1)m+1$ & \dots & $(n-1)m+(m-2)$ & $(n-1)m+(m-1)$
    \end{tabular}
\end{table}

which consists of integers from $0$ to $mn-1$ forming a complete residue system (mod $mn$).

Clearly, each row of the array forms a complete residue system (mod $m$) and all the elements of any column are congruent to each other (mod $m$). Now there are two types of columns: $\phi(m)$ columns containing only co-primes to $m$ and the rest containing none of them. %(\textcolor{blue}{lähde?})

Now consider the co-prime columns. Every column forms a complete residue system (mod $n$)
%(\textcolor{blue}{LeVeque: thm. 3.5, chp. 3.2, p. 52})
\cite{LeVeque}, meaning each includes $\phi(n)$ elements co-prime to $n$. Counting $\phi(n)$ elements from all the $\phi(m)$ columns we get in total $\phi(m)\phi(n)$ numbers that are relatively prime to both $m$ and $n$.

On the other hand, since $gcd(m,n)=1$, an integer $k$ is co-prime to $mn$ if and only if both $gcd(m,k)=1$ and $gcd(n,k)=1$ are true. Hence there are $\phi(m)\phi(n)$ numbers relatively prime to $mn$. Thus by definition $\phi(mn)=\phi(m)\phi(n)$.

The case $m=1$ or $n=1$ is trivial, since $\phi(1)=1$ and hence $\phi(mn)=\phi(m)\phi(n)$.

\end{proof}

\end{theorem}

The definition of Euler's totient function is rather verbal, making it somewhat inconvenient to handle in computation. Fortunately, there is a simple formula to calculate the value as a product involving primes that divide $n$.

\begin{theorem}
Euler's product formula
\label{thm:product}

%\textcolor{blue}{Hardy-Wright: chapter 5.5, theorem 62, p. 64}\\
%\textcolor{red}{Oisko jossain parempi lähde}

\begin{equation*}
    \phi(n) = n \prod_{p \,\vert\, n} \left(1 - \frac{1}{p}\right)
\end{equation*}

where $\prod_{p \,\vert\, n} \left(1 - \frac{1}{p}\right)$ stands for the product over those distinct primes that divide $n$.

\begin{proof}

Assume first that $n = p^k$, where $p\in \Prime$. Integers $x$, for which $gdc(p^k,x)>1$, are exactly the numbers $x=mp$ where $m\in \{1,2,\dots,p^{k-1}\}$.

Hence
\begin{equation*}
    \phi(n)=\phi(p^k)=p^k-p^{k-1}=p^k-\frac{p^k}{p}=\left(1-\frac{1}{p}\right)p^k=\left(1-\frac{1}{p}\right)n.
\end{equation*}

Then, in the general case, assume $n=p_1^{k_1} p_2^{k_2} \cdot \cdot \cdot p_r^{k_r}=\prod_{i=1}^r p_i^{k_i}$, where $p_1,p_2,\dots,p_r$ are distinct primes that divide $n$ and $k_1,k_2,\dots,k_r$ their powers respectively. 

% Toi "their powers" ei oo hyvä, joku kertaluku tai muu käsite pitäis kaivaa siihen

Now, since $\phi$ is a multiplicative function
\begin{align*}
    \phi(n) & = \phi(p_1^{k_1} p_1^{k_1} \cdot \cdot \cdot p_r^{k_r})\\
    & = \phi(p_1^{k_1})\,\phi(p_2^{k_2}) \cdot \cdot \cdot \phi(p_r^{k_r})\\
    & = \left(1-\frac{1}{p_1}\right)p_1^{k_1} \left(1-\frac{1}{p_2}\right)p_2^{k_2} \cdot \cdot \cdot \left(1-\frac{1}{p_r}\right)p_r^{k_r}\\
    & = \prod_{i=1}^r \left(1-\frac{1}{p_i}\right) p_i^{k_i}\\
    & = n \prod_{p \,\vert\, n} \left(1 - \frac{1}{p}\right).
\end{align*}

\end{proof}

\end{theorem}

When it comes to primes, the value of the totient function is easy to deduce. By definition, primes are not divisible by any other number than themself and one, yielding the following lemma.

\begin{lemma}
\label{thm:phiprime}
For every $p \in \Prime$ holds $\phi(p) = p-1$\,.

\begin{proof}

%\textcolor{red}{Vois lisää nopee suoranki todistuksen}

Let $n\in\Prime$. We observe that the only prime that divides $n$ is $n$ itself. Hence by the Euler's product formula
\begin{equation*}
    \phi(n) = n \prod_{p \,\vert\, n} \left(1 - \frac{1}{p}\right) = n\left(1-\frac{1}{n}\right) = n-1\,.
\end{equation*}

\end{proof}

\end{lemma}

\section{Helpful functions and results}
\label{apujutut}

In order to determine the order of growth of the totient function, we must introduce few functions and auxiliary results that are used in the proof of the lower limit. Since all of the results of this chapter serve mainly as tools, we do not give details for some of them.

\subsection{Mertens' theorems}

The most important of the results is the Mertens' third theorem. We still need some lemmas before the proof of the theorem itself. First, we introduce few functions. The $O$-function, defined formally below, describes the behavior of the limit of another function.

\begin{definition}
O-function

Let $x$ be a real variable tending to infinity and $f(x)$ and $g(x)$ be functions of $x$. We say $f(x)=O(g(x))$, if $\vert f(x) \vert < Ag(x)$ for some constant $A$.
\end{definition}

For example $8x=O(x)$ and $\sin{x}=O(1)$. We can also do calculations with the function, for example $O(x)+O(1)=O(x)$ and $O(x)\cdot O(x) = O(x^2)$.

Next up is the $\Lambda$-function, the partial sum of which has some useful properties keeping an eye on the Mertens' theorems.

\begin{definition}
Von Mangoldt $\Lambda$-function

Let $p\in\Prime$ and $k\geq1$.
\begin{equation*}
    \Lambda(n) =
    \begin{cases}
    \log p \quad \text{if } n=p^k\\
    0 \quad\quad\,\,\, \text{otherwise}\,.
    \end{cases}
\end{equation*}
\end{definition}

\begin{theorem}
\label{thm:lambdaf}
\begin{equation*}
    \sum_{d\,\vert\, n} \Lambda(d) = \log n\,.
\end{equation*}
\begin{proof}
Let us denote $n=\prod p^k$. Now, by definition, we have
\begin{equation*}
    \sum_{d\,\vert\, n} \Lambda(d) = \sum_{p^k\,\vert\, n} \log p\,.
\end{equation*}
We notice that as the sum runs through all combinations of primes $p$ and positive integer powers $k$ such that $p^k\,\vert\, n$, each $\log p$ occurs $k$ times. Hence
\begin{equation*}
    \sum_{p^k\,\vert\, n} \log p = \sum a \log p = \sum \log p^a = \log \prod p^a = \log n\,.
\end{equation*}
\end{proof}
\end{theorem}

\begin{lemma}
\label{lemma:lambdadd}
\begin{equation*}
    \sum_{n\leq x} \frac{\Lambda(n)}{n} = \log x + O(1)\,.
\end{equation*}
\begin{proof}
First, we have a weak form of so called Stirling's formula \cite{Goldmakher}
\begin{align*}
    \sum_{n\leq x} \log n & = \int_1^x \log t\,d\lfloor t\rfloor\\
    & = \lfloor x\rfloor\log x - \int_1^x \frac{\lfloor t \rfloor}{t}\,dt\\
    & = x\log x - \{x\}\log x - x + 1 + \int_1^x \frac{\{t\}}{t}\,dt\\
    & = x\log x - x + O(\log x)\,.
\end{align*}

On the other hand, by Theorem \ref{thm:lambdaf} we can deduce
\begin{equation*}
    \sum_{n\leq x} \log n = \sum_{n\leq x} \sum_{d\,\vert\, n} \Lambda(d) = \sum_{d\leq x} \Lambda(d)\,\left\lfloor \frac{x}{d}\right\rfloor = x\cdot\sum_{d\leq x} \frac{\Lambda(d)}{d} + O(\psi(x))\,,
\end{equation*}
where $\psi(x)=\sum_{d\leq x} \Lambda(d)$ as defined before.

Now we have
\begin{equation*}
    x\log x - x + O(\log x) = x\cdot\sum_{d\leq x} \frac{\Lambda(d)}{d} + O(\psi(x))
\end{equation*}
and when we divide the equation by $x$, we get the result
\begin{equation*}
    \log x + O(1) = \sum_{d\leq x} \frac{\Lambda(d)}{d}\,,
\end{equation*}
since $\frac{O(\psi(x))}{x}=O(1)$ by Lemma \ref{lemma:cheb}.
\end{proof}
\end{lemma}

It is sometimes useful to denote the partial sum of the $\Lambda$-function with another function, so called $\psi$-function. Let us also introduce another similar function, the $\vartheta$-function, as we will need it too.

\begin{definition}
Chebyshev functions $\vartheta$ and $\psi$

\begin{align*}
     \vartheta(x) & = \sum_{p\leq x} \log p = \log \prod_{p\leq x} p\\
    \psi(x) & = \sum_{p^m\leq x} \log p = \sum_{n\leq x} \Lambda(n)
\end{align*}
\end{definition}

\begin{lemma}
\label{lemma:cheb}
\begin{equation*}
    \psi(x) = O(x) \quad\text{and}\quad \vartheta(x)=O(x)
\end{equation*}
i.e.
\begin{equation*}
    \psi(x) < Ax \quad\text{and}\quad \vartheta(x) < Ax
\end{equation*}
for some constant A.
\begin{proof}
This is proved in \textit{An introduction to the Theory of Numbers} \cite{HardyWright} as Theorem 414 (ch. 22.2, p. 453).
\end{proof}
\end{lemma}

The following lemma, so called Abel's partial summation formula, presents a convenient way to combine a partial sum of a sequence and a continuous function, which will be useful later.

\begin{lemma}
\label{lemma:abelsum}
Abel's partial summation formula \emph{\cite{HardyWright}}

If $c_1, c_2,\dots$ is a sequence of real numbers such that $c_i=0$ for $i<2$ and
\begin{equation*}
    C(t) = \sum_{n\leq t} c_n
\end{equation*}
and $f(t)$ has continuous derivative for $t \geq 2 \in \R$, then
\begin{equation*}
    \sum_{n\leq x} c_n\,f(n) = C(x)\,f(\lfloor x\rfloor)-\int_2^x C(t)\,f'(t)\,dt\,.
\end{equation*}

\begin{proof}
First we notice that $C(n) = C(t)$ and $f(n)=f(\lfloor t\rfloor)$, when $n\leq t < n+1$.

We have
\begin{align*}
    \sum_{n\leq x} c_n\,f(n) & = c_1\,f(1) + c_2\,f(2) + \dots + c_n\,f(n)\\
    & =  C(1)\,f(1) + (C(2)-C(1))\,f(2) + \dots + (C(n)-C(n-1))\,f(n)\\
    & = C(1)\,(f(1)-f(2)) + C(2)\,(f(2)-f(3)) + \dots\\
    & \quad + C(n-1)\,(f(n-1)-f(n)) + C(n)\,f(n)\\
    & = \sum_{n\leq x-1} C(n)\,(f(n)-f(n+1)) + \underbrace{C(n)\,f(n)}_\text{$C(x)\,f(\lfloor x\rfloor)$}\,.
\end{align*}

On the other hand, since $f(t)$ is continuously differentiable when $t\geq2$ and $C(t)=0$ elsewhere, we have
\begin{equation*}
     C(n)\,(f(n)-f(n+1)) = \int_{n+1}^n C(t)\,f'(t)\,dt = -\int_n^{n+1} C(t)\,f'(t)\,dt\,.
\end{equation*}
Finally, by combining these we get
\begin{align*}
    \sum_{n\leq x} c_n\,f(n) & = C(x)\,f(\lfloor x\rfloor) + \sum_{n\leq x-1} C(n)\,(f(n)-f(n+1))\\
    & = C(x)\,f(\lfloor x\rfloor)-\int_2^x C(t)\,f'(t)\,dt\,.
\end{align*}

\end{proof}
\end{lemma}

Next we present two of Mertens' theorems. The third theorem is the one actually used later on, however, we settle for thoroughly proving only the second. That is, the proof of the third (the part where a mysterious constant $\gamma$ pops up) would unfortunately steal too much space from the actual subject of this thesis.

\begin{theorem}
Mertens' second theorem \emph{\cite{Goldmakher}}
%(\textcolor{blue}{Hardy-Wright: thm. 429, chp. 22.8, p. 466})
\label{thm:mertens32}

For some constant $B$,
\begin{equation*}
    %\lim_{n\rightarrow\infty} \left(\sum_{p\leq n}\left(\frac{1}{p}\right) - \log\log n + B\right) = 0\,,
    \sum_{p\leq n}\frac{1}{p} = \log\log n + B + O\left(\frac{1}{\log n}\right)\,.
\end{equation*}

\begin{proof}
We observe that
\begin{align*}
    \sum_{d\leq n} \frac{\Lambda(d)}{d} & = \sum_k \sum_{p^k\leq n} \frac{\log p}{p^k}\\
    & = \sum_{p\leq n} \frac{\log p}{p} + \sum_{p\leq \sqrt{n}} \frac{\log p}{p^2} + \sum_{p\leq \sqrt[3]{n}} \frac{\log p}{p^3} + \dots\\
    & < \sum_{p\leq n} \frac{\log p}{p} + \sum_p \left(\frac{1}{p^2}+\frac{1}{p^3}+\dots\right)\,\log p\\
    & = \sum_{p\leq n} \frac{\log p}{p} + \sum_p \frac{\log p}{p(p-1)}\\
    & < \sum_{p\leq n} \frac{\log p}{p} + \underbrace{ \sum_{m\geq2}^\infty \frac{\log m}{m(m-1)}}_\text{$=:$ constant $A$}\,.
\end{align*}
Since $A$ is a constant, we get
\begin{equation*}
    \sum_{p\leq n} \frac{\log p}{p} = \sum_{d\leq n} \frac{\Lambda(d)}{d} + O(1) = \log n + O(1)
\end{equation*}
by Lemma \ref{lemma:lambdadd}.

Let us then apply the Abel's partial summation formula (Lemma \ref{lemma:abelsum}) with the sequence $(c_k)$ such that $c_p=\frac{\log p}{p}$ with prime indices and $c_k=0$ otherwise. We then have
\begin{equation*}
    C(n) = \sum_{k\leq n} c_k = \sum_{p\leq n} c_p = \sum_{p\leq n} \frac{\log p}{p}\,.
\end{equation*}

Let $f(t)=\frac{1}{\log t}$. Now by the formula we get
\begin{align*}
    \sum_{p\leq n} \frac{1}{p} & = \sum_{p\leq n} c_p\,f(p) = \sum_{k\leq n} c_k\,f(k)\\
    & = \frac{C(n)}{\log n} + \int_2^n \frac{C(t)}{t\log^2 t}\,dt\\
    & = \frac{\log n + O(1)}{\log n} + \int_2^n \frac{\log t + O(1)}{t\log^2 t}\,dt\\
    & = 1 + \frac{O(1)}{\log n} + \int_2^n \frac{dt}{t \log t} + \int_2^n \frac{O(1)}{t \log^2 t}\,dt\\
    & = 1 + O\left(\frac{1}{\log n}\right) + \log\log n - \log\log 2 + \int_2^\infty \frac{O(1)}{t \log^2 t}\,dt - \int_n^\infty \frac{O(1)}{t \log^2 t}\,dt\\
    & = \log\log n + \underbrace{\left(\int_2^\infty \frac{O(1)}{t \log^2 t}\,dt + 1 - \log \log 2\right)}_\text{$=:$ constant $B$} + O\left(\frac{1}{\log n}\right)\,,
\end{align*}
proving the claim.

\end{proof}

\end{theorem}

\begin{theorem}
Mertens' third theorem \emph{\cite{HardyWright}}
%(\textcolor{blue}{Hardy-Wright: thm. 429, chp. 22.8, p. 466})
\label{thm:mertens3}

\begin{equation*}
    e^{-\gamma} = \log n \prod_{p\le n} \left(1-\frac{1}{p}\right)\,\left(1+O\left(\frac{1}{\log n}\right)\right)
\end{equation*}
and especially
\begin{equation*}
    \lim_{n \rightarrow \infty} \log n \prod_{p\leq n} \left(1-\frac{1}{p}\right) = e^{-\gamma}\,,
\end{equation*}

where $\gamma$ is the Euler-Mascheroni constant.

\begin{proof}
To reach the form of Mertens' third theorem, it is shown in \textit{An Introduction to the Theory of Numbers} \cite{HardyWright} as Theorem 428 (ch. 22.8, p. 466) that
\begin{equation*}
    B = \gamma + \sum_{p\leq n} \left(\log \left(1-\frac{1}{p}\right)+\frac{1}{p}\right)\,,
\end{equation*}
where $\gamma$ is the Euler-Masheroni constant.

Let us take the value of $B$ as given and deduce
\begin{align*}
    \sum_{p\leq n} \frac{1}{p} & = \log\log n + B + O\left(\frac{1}{\log n}\right)\\
    & = \log\log n + \gamma + \sum_{p\leq n} \left(\log \left(1-\frac{1}{p}\right)+\frac{1}{p}\right) + O\left(\frac{1}{\log n}\right)
\end{align*}
or equivalently
\begin{align*}
    0 & = \log\log n + \gamma + \sum_{p\leq n} \log \left(1-\frac{1}{p}\right) + O\left(\frac{1}{\log n}\right)\\
    & = \log\log n + \gamma + \log \prod_{p\leq n} \left(1-\frac{1}{p}\right) + O\left(\frac{1}{\log n}\right)
\end{align*}
Thus
\begin{equation*}
    1 = \log n \cdot e^\gamma \cdot \prod_{p\leq n} \left(1-\frac{1}{p}\right) \cdot \left(1+O\left(\frac{1}{\log n}\right)\right)
\end{equation*}
and finally
\begin{equation*}
    e^{-\gamma} = \log n \prod_{p\le n} \left(1-\frac{1}{p}\right) \left(1+O\left(\frac{1}{\log n}\right)\right)
\end{equation*}

Hence
\begin{equation*}
    \lim_{n \rightarrow \infty} \log n \prod_{p\leq n} \left(1-\frac{1}{p}\right) = e^{-\gamma}\,.
\end{equation*}
\end{proof}

\end{theorem}

\begin{remark}
About the Euler-Mascheroni constant. \emph{\cite{gamma}}

The Euler-Mascheroni constant $\gamma$ equals the limit of the difference of the harmonic series and natural logarithm,
\begin{equation*}
    \gamma=\lim_{n\rightarrow\infty} \left(\sum_{k=1}^n \frac{1}{k} - \log n\right) \approx 0,57721566\,.
\end{equation*}
The constant appears in the Mertens' third theorem and later in the lower limit of the totient function.

\end{remark}

\subsection{Functions $\sigma$ and $\zeta$}

The $\sigma$-function is also a number theoretical and quite closely related to the $\phi$-function itself, as we will soon see. The value of $\sigma(n)$ is the sum of the divisors of $n$, or formally defined as below.

\begin{definition}
The sigma function
%(\textcolor{blue}{Hardy-Wright: chp. 16.7, p. 310})

\begin{equation*}
    \sigma(n)=\sum_{d\,\vert\, n} d
\end{equation*}
\end{definition}

\begin{lemma}
\label{lemma:sigma}
Let $n=p_1^{k_1}p_2^{k_2}\cdots p_r^{k_r}$ be the prime factorization of $n$, where $p_1,p_2,\dots,p_r$ are distinct primes. Then
\begin{equation*}
    \sigma(n)=\prod_{i=1}^r \frac{p_i^{k_i+1}-1}{p_i-1}\,,
\end{equation*}
also denoted by
\begin{equation*}
    \sigma(n) = \prod_{p\,\vert\, n} \frac{p^{k+1}-1}{p-1}\,.
\end{equation*}

\begin{proof}
This is proved in \textit{An introduction to the Theory of Numbers} \cite{HardyWright} as Theorem 275 (ch. 16.7, p. 311).
%\textcolor{blue}{Hardy-Wright: thm. 275, chp. 16.7, p. 311 }.
\end{proof}
\end{lemma}

\begin{lemma}
\label{thm:sigmafii}
\begin{equation*}
    \frac{\phi(n)\,\sigma(n)}{n^2}<1
\end{equation*}

\begin{proof}
By the Euler's product formula and Lemma \ref{lemma:sigma} we get
\begin{align*}
    \phi(n)\,\sigma(n) & = n\prod_{p\,\vert\, n}\left(1-\frac{1}{p}\right) \cdot \prod_{p\,\vert\, n} \frac{p^{k+1}-1}{p-1}\\
    & = n\prod_{p\,\vert\, n}\left(1-\frac{1}{p}\right) \cdot \prod_{p\,\vert\, n} p^k \cdot \prod_{p\,\vert\, n} \frac{p-\frac{1}{p^k}}{p-1}\\
    & = n\prod_{p\,\vert\, n}\left(1-\frac{1}{p}\right) \cdot n \prod_{p\,\vert\, n}\frac{1-\frac{1}{p^{k+1}}}{1-\frac{1}{p}}\\
    & = n^2 \prod_{p\,\vert\, n} \left( \frac{1-\frac{1}{p^{k+1}}}{1-\frac{1}{p}} - \frac{1-\frac{1}{p^{k+1}}}{p-1}\right)\\
    & = n^2 \prod_{p\,\vert\, n} \frac{p-1-\frac{1}{p^k}+\frac{1}{p^{k+1}}}{p-1}\\
    & = n^2 \prod_{p\,\vert\, n} \frac{p-1-(p-1)\frac{1}{p^{k+1}})}{p-1}\\
    & = n^2 \prod_{p\,\vert\, n} \left(1-\frac{1}{p^{k+1}}\right) < n^2\,.
\end{align*}
Equivivalently
\begin{equation*}
    \frac{\phi(n)\,\sigma(n)}{n^2}<1\,.
\end{equation*}
\end{proof}
\end{lemma}

We also need briefly the $\zeta$-function later.

\begin{definition}
Riemann $\zeta$-function \emph{\cite{HardyWright}}
%(\textcolor{blue}{Hardy-Wright: chp. 17.2, p. 320})

\begin{equation*}
    \zeta(s)=\sum_{n=1}^\infty \frac{1}{n^s}\,,
\end{equation*}
where $s\in\R$.
\end{definition}

\begin{lemma}
\label{lemma:zeta}
For all $s>1\in\R$, 
\begin{equation*}
    \zeta(s)=\prod_p \frac{1}{1-\frac{1}{p^s}}\,.
\end{equation*}
\begin{proof}
This is proved in \textit{An introduction to the Theory of Numbers} \cite{HardyWright} as Theorem 280 (ch. 17.2, p. 320).
%\textcolor{blue}{Hardy-Wright: thm. 280, chp. 17.2, p. 320}.
\end{proof}
\end{lemma}

\section{The limits of Euler's totient function}

Finally reaching the point, in which we are equipped to start getting into the order of the totient function, let us still consider, why it is interesting at all. As pondered before, the Euler's product formula gives the totient function a more computable form. However, using it requires factorization of $n$, which still makes it difficult to estimate the size of $\phi(n)$ as $n$ gets bigger.

Let us amuse ourselves with a quick example.

\begin{example}
\emph{\cite{Pomerance}} Let $n = 2^p - 1 \in \Prime$ be so called Mersenne prime, meaning also $p \in \Prime$. By Theorem \ref{thm:phiprime} we know $\phi(n) = n - 1$. On the other hand, from Euler's product formula follows that $\phi(n+1) = \phi(2^p) = 2^p(1-\frac{1}{2}) = \frac{2^p}{2} = \frac{n+1}{2}$.

Now we see that while $n$ and $n+1$ differ from each other only insignificantly, $\phi(n+1)$ is half the size of $\phi(n)$.
\end{example}


All this in our mind, next we will prove the exact limits of the totient function, starting with the fairly obvious upper limit and then diving into a detailed proof of the lower limit.

\subsection{Upper limit of Euler's totient function}

The maximum value of $\phi(n)$ given $n$ is easy to define with Theorem \ref{thm:phiprime}.

\begin{theorem}
Upper limit of the totient function \emph{\cite{HardyWright}}
%(\textcolor{blue}{Hardy-Wright: thm. 326, chp. 18.4, p. 352})

For every $n \geq 2$ holds $\phi(n) \leq n-1$ and
\begin{equation*}
    \limsup{\frac{\phi(n)}{n}} = 1\,.
\end{equation*}

\begin{proof}

By definition, $\phi(n) \leq n$ because there are $n$ elements in the set $\{1,2,\dots,n\}$. Also, for every $n \geq 2$ holds $gcd(n,n) = n \neq 1$. Thus, $\phi(n) \leq n-1$.

On the other hand, according to Theorem \ref{thm:phiprime}, $\phi(p) = p-1$ for every $p\in\Prime$.
% TÄÄ ON VÄÄRIN SANOA NÄIN
% Because there are infinitely many primes, this means that $n-1$ is, in fact, the limit superior of Euler's totient function.
Now, because there are infinitely many primes %(\textcolor{blue}{lähde?}),
\begin{equation*}
    \limsup{\frac{\phi(n)}{n}} = \lim \frac{n-1}{n} = 1\,.
\end{equation*}

\end{proof}

\end{theorem}

\subsection{Lower limit of Euler's totient function}

How small $\phi(n)$ can be as $n$ grows, is much less trivial a question to answer. However, it can be shown that the value of $\phi(n)$ is proportional to $\frac{n}{\log\log n}$. The rest of this paper will cover the proof of the exact limit inferior of the totient function, following the proof of Theorem $328$
in \textit{An Introduction to the Theory of Numbers} \cite{HardyWright} (ch. 22.9, p. 469).

\begin{theorem}
Lower limit of the totient function \emph{\cite{HardyWright}}
%(\textcolor{blue}{Hardy-Wright: thm. 328, chp. 18.4, p. 352})

\begin{equation*}
    \liminf{\frac{\phi(n)\,\log\log n}{n}}=e^{-\gamma}\,,
\end{equation*}
where $\gamma$ is the Euler-Mascheroni constant.

\begin{proof}

Let us prove the claim by showing $\liminf{f(n)} = 1$, when
\begin{equation*}
    f(n)= \frac{\phi(n)\,e^\gamma \log\log n}{n}\,.
\end{equation*}
%and $\gamma$ is the Euler-Mascheroni constant.

The proof is based on finding functions $F_1(t)$ and $F_2(t)$, the limits of which are both $\lim_{t\rightarrow \infty} F_1(t) = 1$ and $\lim_{t\rightarrow \infty} F_2(t) = 1$. First we show that 
\begin{equation}
\label{eq:first}
    f(n) \geq F_1(\log n)\text{ for all }n\geq 3
\end{equation}
and in the second part that
\begin{equation}
\label{eq:second}
    f(n_j) \leq \frac{1}{F_2(j)}\text{ for some infinite increasing sequence }n_2, n_3,\dots
\end{equation}

Let $p_1,p_2,\dots,p_{r-\rho} \leq \log n$ and $p_{r-\rho+1},\dots,p_r > \log n$ be the prime factors of $n$. In other words, the number $n$ has $r$ prime factors, $\rho$ of which are greater than $\log n$.

Now
\begin{equation*}
    (\log n)^\rho < p_{r-\rho+1} \cdot p_{r-\rho+2} \cdots p_r \leq n\,,
\end{equation*}
which yields
\begin{equation*}
    \rho < \frac{\log n}{\log\log n}\,.
\end{equation*}

Thus, there are less than $\frac{\log n}{\log\log n}$ prime factors greater than $\log n$.

By the Euler's product formula (Theorem \ref{thm:product})
\begin{align*}
    \frac{\phi(n)}{n} & = \prod_{i=1}^r\left(1-\frac{1}{p_i}\right)\\
    & = \prod_{i=1}^{r-\rho}\left(1-\frac{1}{p_i}\right) \prod_{i=r-\rho+1}^r\left(1-\frac{1}{p_i}\right)\\
    & = \prod_{p\leq\log n}\left(1-\frac{1}{p}\right) \prod_{p>\log n}\left(1-\frac{1}{p}\right)\\
    & \geq \left(1-\frac{1}{\log n}\right)^\rho \prod_{p\leq\log n}\left(1-\frac{1}{p}\right) \\
    & > \left(1-\frac{1}{\log n}\right)^\frac{\log n}{\log \log n} \prod_{p\leq\log n}\left(1-\frac{1}{p}\right)\,.
\end{align*}

Hence, we can define
\begin{equation*}
    F_1(t)=e^\gamma \log t \left(1-\frac{1}{t}\right)^\frac{t}{\log t} \prod_{p\leq t} \left(1-\frac{1}{p}\right)\,,
\end{equation*}
because by the inequality above
\begin{align*}
    F_1(\log n) & = e^\gamma \log \log n \left(1-\frac{1}{\log n}\right)^\frac{\log n}{\log \log n} \prod_{p\leq \log n} \left(1-\frac{1}{p}\right)\\
    & \leq \frac{\phi(n)}{n} e^\gamma \log\log n = f(n)
\end{align*}
and by the Mertens' third theorem (Theorem \ref{thm:mertens3})
\begin{align*}
    \lim_{t \rightarrow \infty} F_1(t) & = \lim_{t \rightarrow \infty} e^\gamma \log t \left(1-\frac{1}{t}\right)^\frac{t}{\log t} \prod_{p\leq t} \left(1-\frac{1}{p}\right)\\
    & = \lim_{t \rightarrow \infty} e^\gamma \left( 1-\frac{1}{t}\right)^\frac{t}{\log t} \left(\log t \prod_{p\leq t} \left(1-\frac{1}{p}\right) \right)\\
    & = \lim_{t \rightarrow \infty} e^\gamma \left( 1-\frac{1}{t}\right)^\frac{t}{\log t} e^{-\gamma}\\
    & = \lim_{t \rightarrow \infty} \left( 1-\frac{1}{t}\right)^\frac{t}{\log t}\\
    & = 1\,.
\end{align*}

Now we have proved the part (\ref{eq:first}) and showed that
$\liminf{f(n)}\geq 1$.

Next, to prove the part (\ref{eq:second}), let us define
\begin{equation*}
    g(n)=\frac{\sigma(n)}{n\,e^\gamma \log\log n}
\end{equation*}
and show that $g(n_j) \geq F_2(j)$ for an infinite increasing sequence $n_2,n_3\dots$. The desired result will follow from Theorem \ref{thm:sigmafii}.

Let
\begin{equation*}
    n_j=\prod_{p\leq e^j} p^j\text{, where } j\geq 2\,.
\end{equation*}

By the Lemma \ref{lemma:cheb}
\begin{equation*}
    \log n_j = \log \prod_{p\leq e^j} p^j = j \log \prod_{p\leq e^j} p = j\,\vartheta(e^j) \leq A\,j\,e^j\,,
\end{equation*}
where $A$ is a real constant.

Hence
\begin{equation*}
\label{eq:lognj}
    \log \log n_j = \log Aje^j = \log A + \log j + \log e^j = \log A + \log j + j\,.
\end{equation*}

Since $n_j$ is the product of all primes smaller than $e^j$ to the power of $j$, by the Lemma \ref{lemma:sigma} we have
\begin{equation*}
    \sigma(n_j) = \prod_{p\leq e^j} \frac{p^{j+1}-1}{p-1}
\end{equation*}
and
\begin{equation*}
    \frac{\sigma(n_j)}{n_j} = \prod_{p\leq e^j} \frac{p^{j+1}-1}{(p-1)p^j} = \prod_{p\leq e^j} \frac{p^{j+1}\left(1-\frac{1}{p^{j+1}}\right)}{p^{j+1}\left(1-\frac{1}{p}\right)} = \prod_{p\leq e^j} \frac{1-\frac{1}{p^{j+1}}}{1-\frac{1}{p}}\,.
\end{equation*}

Also, by the Lemma \ref{lemma:zeta}
\begin{equation*}
    \prod_{p\leq e^j}\left(1-\frac{1}{p^{j+1}}\right) > \prod \left(1-\frac{1}{p^{j+1}}\right) = \frac{1}{\zeta(j+1)}\,.
\end{equation*}

Now we can define
\begin{equation*}
    F_2(t)=\frac{1}{e^\gamma\,\zeta(t+1)(B+t+\log t)} \prod_{p\leq e^t} \left(\frac{1}{1-\frac{1}{p}}\right)\,,
\end{equation*}
where $B=\log A$ is a suitable real constant.

This is, by combining the results above
\begin{align*}
    F_2(j)& = \frac{1}{e^\gamma\,\zeta(j+1)(B+j+\log j)} \prod_{p\leq e^j} \left(\frac{1}{1-\frac{1}{p}}\right)\\
    & \leq \frac{1}{e^\gamma\,\log \log n_j} \prod_{p\leq e^j} \frac{1-\frac{1}{p^{j+1}}}{1-\frac{1}{p}}\\
    & = \frac{\sigma(n_j)}{n_j\,e^\gamma\,\log \log n_j} = g(n_j)\,.
\end{align*}

By the Mertens' third theorem (Theorem \ref{thm:mertens3})

\begin{equation*}
    \lim_{t \rightarrow \infty} \prod_{p\leq e^t} \left(\frac{1}{1-\frac{1}{p}}\right) = \lim_{t \rightarrow \infty} \frac{1}{\prod_{p\leq e^t} \left(1-\frac{1}{p}\right)} = \left(\frac{e^{-\gamma}}{\log e^t}\right)^{-1} = e^\gamma\,t
\end{equation*}

and since $\zeta(t+1)\rightarrow 1$ when $t\rightarrow\infty$ \cite{HardyWright}, we now have
\begin{align*}
    \lim_{t \rightarrow \infty} F_2(t) & = \lim_{t \rightarrow \infty} \frac{1}{e^\gamma\,\zeta(t+1)(B+t+\log t)} \prod_{p\leq e^t} \left(\frac{1}{1-\frac{1}{p}}\right)\\
    & = \lim_{t \rightarrow \infty} \frac{e^\gamma\,t}{e^\gamma\,\zeta(t+1)(B+t+\log t)}\\
    & = \lim_{t \rightarrow \infty} \frac{t}{\zeta(t+1)(B+t+\log t)}\\
    & = \lim_{t \rightarrow \infty} \frac{t}{B+t+\log t}\\
    & = 1\,.
\end{align*}

By the Theorem \ref{thm:sigmafii}
\begin{equation*}
    f(n)\,g(n) = \frac{\phi(n)\,e^\gamma \log\log n}{n} \cdot \frac{\sigma(n)}{n\,e^\gamma \log\log n} = \frac{\phi(n)\sigma(n)}{n^2}<1
\end{equation*}

and since $g(n_j) \geq F_2(j)$

\begin{equation*}
    f(n_j)\leq \frac{1}{F_2(j)}\,.
\end{equation*}

Thus we have proved the part (\ref{eq:second}) and showed that $\liminf{f(n)\leq 1}$.

Altogether, from the parts (\ref{eq:first}) and (\ref{eq:second}), we get that the limit inferior of $f(n)$ must be
\begin{equation*}
    \liminf{\frac{\phi(n)\,e^\gamma \log\log n}{n}}=\liminf{f(n)}=1\,
\end{equation*}
and equivalently
\begin{equation*}
    \liminf{\frac{\phi(n)\,\log\log n}{n}}=e^{-\gamma}\,.
\end{equation*}
\end{proof}
\end{theorem}

\begin{figure}[!htb]
    \centering
    \includegraphics[width =120mm]{kuva2-philimits.eps}
    \caption{Upper and lower bound of Euler's totient function with small and relatively large $n$. We see that the lower bound only holds when $n\rightarrow\infty$.}
    \label{fig:k2}
\end{figure}

\clearpage
\nocite{*}
\printbibliography

\end{document}
