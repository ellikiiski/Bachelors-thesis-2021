\documentclass{article}
\usepackage[utf8]{inputenc}
\usepackage{lmodern}
\usepackage[T1]{fontenc}
\usepackage[english]{babel}
\usepackage{amsmath}
\usepackage{amsthm}
\usepackage{amssymb}
\usepackage{parskip}
\usepackage[dvipsnames]{xcolor}

\usepackage[backend=biber,style=numeric]{biblatex}
\usepackage{csquotes}
\addbibresource{viitteet.bib}

\title{The order of Euler's totient function}
\author{Elli Kiiski}
\date{2021}

\theoremstyle{definition}
\newtheorem{definition}[subsubsection]{Definition}
\newtheorem{notation}[subsubsection]{Notation}
\newtheorem{lemma}[subsubsection]{Lemma}
\newtheorem{theorem}[subsubsection]{Theorem}

\begin{document}

{\large
Elli Kiiski
\par
\textbf{2021 Kandinalku - The order of Euler's totient function}
}
\vspace{0.5cm}

\section{Introduction}

\textcolor{gray}{Placeholder for some introductory explaining about the subject of this thesis.}

All introduced variables $a, b, c, ...$ are integers, unless stated otherwise. Here the set of natural numbers $\mathbb{N}$ consists of positive integers, meaning $0 \not\in \mathbb{N}$.

\begin{notation}{\emph{Divisibility}}

Let $a$ and $b$ be such that $b$ is divisible by $a$. This is denoted by $a \vert b$.

\end{notation}

\begin{definition}{\emph{Greatest common divisor}}

Let $a \in \mathbb{N}$ and $b \in \mathbb{N}$. It can be shown that there is a unique $d \in \mathbb{N}$ with following properties:

\begin{enumerate}
 \item $d \vert a$ and $d \vert b$
 \item if $c \vert a$ and $c \vert b$, then $c \vert d$
\end{enumerate}

The number $d$ is called the greatest common divisor of $a$ and $b$, denoted by $gcd(a,b) = d$.

However, the existence of the greatest common divisor is non-trivial. Proof can be found in \textit{LeVeque: Fundamentals of Number Theory, chapter 2.1}.

\end{definition}

\begin{definition}{\emph{Prime number}}

Integer $p\in\mathbb{N}$ is a prime, if $p \geq 2$ and for every $k\in\mathbb{N}$ holds that if $k \vert p$ then $k\in{1, p}$. The set of prime numbers is denoted by $\mathbb{P}$.

In other words, all integers greater than $1$, which are only divisible by themself and $1$, are primes.

%Toisin sanoen alkulukuja ovat kaikki lukua $1$ suuremmat luonnolliset luvut, jotka ovat jaollisia vain itsellään ja luvulla $1$.

\end{definition}

\begin{definition}{\emph{Co-prime}}

If $gcd(a,b) = 1$, $a$ and $b$ are called co-primes or relative primes.

\end{definition}

\begin{definition}{\emph{Multiplicative number theoretic function}}

Function $f: \mathbb{N} \rightarrow \mathbb{R}$ is called number theoretic function. It is multiplicative if $f(ab) = f(a)f(b)$ when $gcd(a, b)=1$.

\end{definition}

\section{Euler's totient function and its properties}

Euler's totient function is a multiplicative number theoretic function...

\begin{definition}{\emph{Euler's totient function $\phi: \mathbb{N} \rightarrow \mathbb{N}$}}

It is set that $\phi(1) = 1$. For all $n \geq 2$, $\phi(n)$ is the number of integers $a \in \{1,2,...,n\}$, for which $gcd(a,n) = 1$.

%Toisin sanoen Eulerin $\phi$-funktion arvo luonnollisella luvulla $n$ on sitä pienempien luonnolisten lukujen määrä, jotka ovat alkulukuja sen suhteen.

\end{definition}

That is, the value of the totient function at $n \in \mathbb{N}$ is the number of co-primes of $n$ smaller than it.

\begin{theorem}
Euler's totient function is multuplicative.
\begin{proof}
\textcolor{gray}{Placeholder for proof.}
\end{proof}
\end{theorem}

\begin{theorem}{\emph{Euler's product formula}}
\label{thm:product}

\begin{equation*}
    \phi(n) = n \prod_{p \vert n} (1 - \frac{1}{p})
\end{equation*}

where $\prod_{p \vert n} (1 - \frac{1}{p})$ means the product over \emph{distinct} primes that divide $n$.

\begin{proof}

Assume first that $n = p^k$, where $p\in \mathbb{P}$. Now for every $x$, for which $gdc(p^k,x)>1$, holds $x=mp^{k-1}$ for some $m\in \{1,2,...,p^{k-1}\}$.

% Siis lukujen x, joille gcd(n,x) määrä on p^{k-1}

Hence
\begin{equation*}
    \phi(n)=\phi(p^k)=p^k-p^{k-1}=p^k-\frac{p^k}{p}=\left(1-\frac{1}{p}\right)p^k=\left(1-\frac{1}{p}\right)n.
\end{equation*}

Then, in the general case, assume $n=p_1^{k_1} p_2^{k_2} \cdot \cdot \cdot p_r^{k_r}=\prod_{i=1}^r p_i^{k_i}$, where $p_1,p_2,...,p_r$ are distinct primes that divide $n$ and $k_1,k_2,...,k_r$ their powers respectively. 

% Toi "their powers" ei oo hyvä, joku kertaluku tai muu käsite pitäis kaivaa siihen

Now, since $\phi$ is a multiplicative function
\begin{align*}
    \phi(n) & = \phi(p_1^{k_1} p_1^{k_1} \cdot \cdot \cdot p_r^{k_r})\\
    & = \phi(p_1^{k_1})\,\phi(p_2^{k_2}) \cdot \cdot \cdot \phi(p_r^{k_r})\\
    & = \left(1-\frac{1}{p_1}\right)p_1^{k_1} \left(1-\frac{1}{p_2}\right)p_2^{k_2} \cdot \cdot \cdot \left(1-\frac{1}{p_r}\right)p_r^{k_r}\\
    & = \prod_{i=1}^r \left(1-\frac{1}{p_i}\right) p_i^{k_i}\\
    & = n \prod_{p \vert n} \left(1 - \frac{1}{p}\right).
\end{align*}

\end{proof}

\end{theorem}

\begin{theorem}{\emph{Totient function and primes}}
\label{thm:phiprime}
\begin{equation*}
    \text{For every } p \in \mathbb{P} \text{ holds } \phi(p) = p-1\,.
\end{equation*}

\begin{proof}

Let $n\in\mathbb{P}$. Now the only prime that divides $n$ is $n$ itself. Hence by the Euler's product formula
\begin{equation*}
    \phi(n) = n \prod_{p \vert n} (1 - \frac{1}{p}) = n\left(1-\frac{1}{n}\right) = n-1\,.
\end{equation*}

\end{proof}

\end{theorem}

\section{Merten's theorem and other lemmas}

Before starting with the order of the totient function, we must introduce few theorems that are used in the proof of the lower limit.

\begin{theorem}{\emph{Merten's (third) theorem}}
\label{thm:mertens}
\begin{equation*}
    \lim_{n \rightarrow \infty} \log n \prod_{p\leq n} \left(1-\frac{1}{p}\right) = e^{-\gamma}
\end{equation*}

where $\gamma$ is the Euler's constant.

\begin{proof}

\textcolor{gray}{Placeholder for a sketch of the proof or maybe even the whole proof.}

\end{proof}

\end{theorem}

\begin{definition}{\emph{The $\sigma$-function}}

\begin{equation*}
    \sigma(n)=\sum_{d\vert n} d\,,
\end{equation*}

meaning $\sigma(n)$ is the sum of the divisors of $n$.
\end{definition}

\begin{lemma}
\textcolor{gray}{Placeholder for a lemma that would explain some things in the proof of the lower limit...}
\end{lemma}


\begin{definition}{\emph{Chebyshev function}}
\begin{equation*}
    \vartheta(x)=\sum_{p\leq x} \log p = \log \prod_{p\leq x} p\,,
\end{equation*}
where $x\in\mathbb{R}$ and $p\in\mathbb{P}$.

\end{definition}

\begin{lemma}
\label{lemma:vartheta}
For the function $\vartheta(x)$ holds
\begin{equation*}
    \vartheta(x) < Ax\,,
\end{equation*}
where $x\geq2\in\mathbb{R}$, $A$ is a real constant.

\begin{proof}
Theorem 414 in \textit{Hardy \& Wright: Introduction to the Theory of Numbers}.
\end{proof}
\end{lemma}

\begin{definition}{\emph{Riemann zeta-function}}

\begin{equation*}
    \zeta(s)=\sum_{n=1}^\infty \frac{1}{n^s}\,,
\end{equation*}
where $s\in\mathbb{R}$.
\end{definition}

\begin{lemma}
For all $s>1\in\mathbb{R}$, 
\begin{equation*}
    \zeta(s)=\prod_p \frac{1}{1-\frac{1}{p^s}}\,.
\end{equation*}
\begin{proof}
Theorem 280 in \textit{Hardy \& Wright: Introduction to the Theory of Numbers}.
\end{proof}
\end{lemma}

\section{The limits of Euler's totient function}

As shown in previous chapter, there is an exact formula for the rather verbally defined totient function $\phi(n)$. Though, using it requires factorization of $n$, which seems to cause the difficulty to estimate its size as $n$ gets bigger.

For example, let $n = 2^p - 1 \in \mathbb{P}$ be so called Mersenne prime, meaning also $p \in \mathbb{P}$. By theorem \ref{thm:phiprime} we know $\phi(n) = n - 1$. On the other hand, from Euler's product formula follows that $\phi(n+1) = \phi(2^p) = 2^p(1-\frac{1}{2}) = \frac{2^p}{2} = \frac{n+1}{2}$. Now we see that while $n$ and $n+1$ differ from each other only insignificantly, $\phi(n+1)$ is half the size of $\phi(n)$.

\subsection{Upper limit of Euler's totient function}

The maximum value of $\phi(n)$ given $n$ is easy to define by the theorem \ref{thm:phiprime}.

\begin{theorem}{\emph{Upper limit of the totient function}}

For every $n \geq 2$
\begin{equation*}
    \phi(n) \leq n-1\,.
\end{equation*}

\begin{proof}

By definition $\phi(n) \leq n$ beacause there are $n$ elements in the set $\{1,2,...,n\}$. Also, for every $n \geq 2$ holds $gcd(n,n) = n \neq 1$. Thus, $\phi(n) \leq n-1$.

On the other hand, according to theorem \ref{thm:phiprime}, $\phi(p) = p-1$ for every $p\in\mathbb{P}$.
% TÄÄ ON VÄÄRIN SANOA NÄIN
% Because there are infinitely many primes, this means that $n-1$ is, in fact, the limit superior of Euler's totient function.
Now, because there are infinitely many primes,
\begin{equation*}
    \limsup{\frac{\phi(n)}{n}} = \lim \frac{n-1}{n} = 1\,.
\end{equation*}

\textcolor{red}{Onkohan yllä oleva ensimmäinen yhtäsuuruusmerkki ihan legit? Myös: pitääkö infinitely many primes perustella?}

\end{proof}

\end{theorem}

\subsection{Lower limit of Euler's totient function}

How small $\phi(n)$ can be as $n$ grows, is much less trivial a question to answer. However, the following lower limit exists.

\begin{theorem}{\emph{Lower limit of the totient function}}

\begin{equation*}
    \liminf{\frac{\phi(n)}{n}}=\frac{1}{e^\gamma\log\log n}\,,
\end{equation*}
where $\gamma$ is the Euler's constant.

\begin{proof}

Let's prove the claim by showing $\liminf{f(n)} = 1$, when
\begin{equation*}
    f(n)= \frac{\phi(n)\,e^\gamma \log\log n}{n}\,,
\end{equation*}
and $\gamma$ is the Euler's constant.

The proof is based on finding two functions $F_1(t)$ and $F_2(t)$, the limits of which are both $\lim_{t\rightarrow \infty} F_1(t) = 1$ and $\lim_{t\rightarrow \infty} F_2(t) = 1$. First we show that 
\begin{equation}
\label{eq:first}
    f(n) \geq F_1(\log n)\text{ for all }n\geq 3
\end{equation}
and in the second part that
\begin{equation}
\label{eq:second}
    f(n_j) \leq \frac{1}{F_2(j)}\text{ for some infinite increasing sequence }n_2, n_3,...
\end{equation}

Let $p_1,p_2,...,p_{r-\rho} \leq \log n$ and $p_{r-\rho+1},...,p_r > \log n$ be the prime factors of $n$. In other words, the number $n$ has $r$ prime factors, $\rho$ of which are greater than $\log n$.

Now
\begin{equation*}
    (\log n)^\rho < p_{r-\rho+1} \cdot p_{r-\rho+2} \cdot \cdot \cdot p_r \leq n\,,
\end{equation*}
which yields
\begin{equation*}
    \rho < \frac{\log n}{\log\log n}\,.
\end{equation*}

Thus, there are less than $\frac{\log n}{\log\log n}$ prime factors greater than $\log n$.

By the Euler's product formula (theorem \ref{thm:product})
\begin{align*}
    \frac{\phi(n)}{n} & = \prod_{i=1}^r\left(1-\frac{1}{p_i}\right)\\
    & = \prod_{i=1}^{r-\rho}\left(1-\frac{1}{p_i}\right) \prod_{i=r-\rho+1}^r\left(1-\frac{1}{p_i}\right)\\
    & \geq \left(\prod_{i=1}^{r-\rho}\left(1-\frac{1}{p_i}\right)\right) \left(1-\frac{1}{\log n}\right)^\rho\\
    & > \left(\prod_{i=1}^{r-\rho}\left(1-\frac{1}{p_i}\right)\right) \left(1-\frac{1}{\log n}\right)^\frac{\log n}{\log \log n}\,.
\end{align*}

Hence, we can define
\begin{equation*}
    F_1(t)=e^\gamma \log t \left(1-\frac{1}{t}\right)^\frac{t}{\log t} \prod_{p\leq t} \left(1-\frac{1}{p}\right)\,,
\end{equation*}
because
\begin{align*}
    F_1(\log n) & = e^\gamma \log \log n \left(1-\frac{1}{\log n}\right)^\frac{\log n}{\log \log n} \prod_{p\leq \log n} \left(1-\frac{1}{p}\right)\\
    & = e^\gamma \log \log n \left(1-\frac{1}{\log n}\right)^\frac{\log n}{\log \log n} \prod_{i=1}^{r-\rho} \left(1-\frac{1}{p}\right)\\
    & \leq \frac{\phi(n)}{n} e^\gamma \log\log n = f(n)\,.
\end{align*}
and by the Merten's third theorem (theorem \ref{thm:mertens})
\begin{align*}
    \lim_{t \rightarrow \infty} F_1(t) & = \lim_{t \rightarrow \infty} e^\gamma \log t \left(1-\frac{1}{t}\right)^\frac{t}{\log t} \prod_{p\leq t} \left(1-\frac{1}{p}\right)\\
    & = \lim_{t \rightarrow \infty} e^\gamma \left( 1-\frac{1}{t}\right)^\frac{t}{\log t} \left(\log t \prod_{p\leq t} \left(1-\frac{1}{p}\right) \right)\\
    & = \lim_{t \rightarrow \infty} e^\gamma \left( 1-\frac{1}{t}\right)^\frac{t}{\log t} e^{-\gamma}\\
    & = \lim_{t \rightarrow \infty} \left( 1-\frac{1}{t}\right)^\frac{t}{\log t}\\
    & = 1\,.
\end{align*}

Now we have proved the part (\ref{eq:first}) and showed that the limit inferior of the function $f(n)$ is greater or equal to $1$.

Next, to prove the part (\ref{eq:second}), let
\begin{equation*}
    n_j=\prod_{p\leq e^j} p^j\text{, where } j\geq 2\,.
\end{equation*}

By the lemma \ref{lemma:vartheta}
\begin{equation*}
    \log n_j = \log \prod_{p\leq e^j} p^j = j \log \prod_{p\leq e^j} p = j\vartheta(e^j) \leq Aje^j\,.
\end{equation*}

Hence
\begin{equation*}
    \log \log n_j = \log Aje^j = \log A + \log j + \log e^j = \log A + \log j + j\,.
\end{equation*}

\textcolor{red}{Sit hei väännetään tää todistus loppuun eiksje}

\end{proof}

\end{theorem}

\end{document}