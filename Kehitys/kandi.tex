\documentclass{article}
\usepackage[utf8]{inputenc}
\usepackage{lmodern}
\usepackage[T1]{fontenc}
\usepackage[english]{babel}
\usepackage{amsmath}
\usepackage{amsthm}
\usepackage{amssymb}
\usepackage{parskip}
\usepackage[dvipsnames]{xcolor}

\usepackage[backend=biber,style=numeric]{biblatex}
\usepackage{csquotes}
\addbibresource{viitteet.bib}

\title{The order of Euler's totient function}
\author{Elli Kiiski}
\date{2021}

\theoremstyle{definition}
\newtheorem{definition}[subsection]{Definition}
\newtheorem{lemma}[subsection]{Lemma}
\newtheorem{theorem}[subsection]{Theorem}

\begin{document}

{\large
Elli Kiiski
\par
\textbf{2021 Kandinalku - The order of Euler's totient function}
}
\vspace{0.5cm}

\section{Introduction}

In number theory Euler's totient function is used pops up when considering divisibility, congruences and residue systems. \textcolor{red}{AAAAAA EN OSAA PUHUAAAA} \textcolor{gray}{Anyway tähän vielä paasausta}

Therefore, it is essential to go through the basic definitions concerning mentioned areas before even introducing the totient function itself.

All introduced variables $a, b, c, ...$ are integers, unless stated otherwise. Here the set of natural numbers $\mathbb{N}$ consists of positive integers, meaning $0 \not\in \mathbb{N}$. Also, the notation $[x]$ is used to denote the integer part of a real number $x$.

\begin{definition}
\textit{Divisibility}

If $b=ka$ for some integer $k$, $b$ is divisible by $a$. This is denoted by $a \,\vert\, b$.

\end{definition}

\begin{theorem}
\textit{Greatest common divisor} \cite{LeVeque}

Let $a \in \mathbb{N}$ and $b \in \mathbb{N}$. There is a unique $d \in \mathbb{N}$ with following properties:

\begin{enumerate}
 \item $d \vert a$ and $d \vert b$
 \item if $c \vert a$ and $c \vert b$, then $c \vert d$
\end{enumerate}

The number $d$ is called the greatest common divisor of $a$ and $b$, denoted by $gcd(a,b) = d$.

\begin{proof}
Theorem is proved in \cite{LeVeque} as theorem 2.1 in chapter 2.1 on page 31.
%\textcolor{blue}{LeVeque: thm 2.1, chp 2.1, p. 31}
\textcolor{red}{Oh god en osaa viitata tällasessa plz help}
\end{proof}
\end{theorem}

\begin{definition}
\textit{Congruence}

Let $m\not=0$. If $m\vert(a-b)$, we say $a$ is congruent to $b$ modulo $m$. It is denoted by $a\equiv b \pmod{m}$.
\end{definition}

\begin{definition}
\textit{Complete residue system (mod m)}

The set $a_0,a_1,\dots,a_{m-1}$ forms a complete residue system if
\begin{equation*}
    a_i \equiv i \pmod{m}
\end{equation*}f
or all $i\in\{0,1,\dots,m-1\}$.
\end{definition}

\begin{definition}
\textit{Prime number}

Integer $p\in\mathbb{N}$ is a prime, if $p \geq 2$ and for every $k\in\mathbb{N}$ holds that if $k \,\vert\, p$ then $k\in\{1, p\}$. The set of prime numbers is denoted by $\mathbb{P}$.

%In other words, all integers greater than $1$, only divisible by themself and $1$, are primes.

\end{definition}

\begin{definition}
\textit{Co-prime}

If $gcd(a,b) = 1$, $a$ and $b$ are called co-primes or relatively primes.
\end{definition}

\begin{definition}{\emph{Multiplicative number theoretic function}}

Function $f\colon \mathbb{N} \rightarrow \mathbb{R}$ is called number theoretic function. It is multiplicative if $f(ab) = f(a)f(b)$ when $gcd(a, b)=1$.

\end{definition}

\section{Euler's totient function and its properties}

Now that we have revised some of the basic concepts in number theory, let us introduce Euler's totient function itself. It is a number theoretic function describing, by how many positive integers a given number is, or to be exact is not, divisible. We start by formally defining the function and then showing a few of its properties.

\begin{definition}
\textit{Euler's totient function $\phi: \mathbb{N} \rightarrow \mathbb{N}$}
%\textcolor{blue}{Hardy-Wright: chapter 5.5, p. 63}

It is set that $\phi(1) = 1$. For all $n \geq 2$, $\phi(n)$ is the number of integers $a \in \{1,2,...,n\}$, for which $gcd(a,n) = 1$.

\end{definition}

%That is, the value of the totient function at $n \in \mathbb{N}$ is the number of co-primes of $n$ smaller than it.

That is, the value of $\phi(n)$ is the number of positive co-primes of $n$ less or equal to $n$.

\begin{theorem}
%\textcolor{blue}{Hardy-Wright: chapter 5.5, theorem 60, p. 64}
Euler's totient function is multiplicative, i.e.
\begin{equation*}
    gcd(m,n)=1 \quad\Rightarrow\quad \phi(mn)=\phi(m)\phi(n)\,.
\end{equation*}

\begin{proof}
Assume $m>1$, $n>1$ and $gcd(m,n)=1$. Consider the array

\begin{table}[!htb]
    \centering
    \begin{tabular}{ccccc}
        $0$ & $1$ & \dots & $m-2$ & $m-1$ \\
        $m$ & $m+1$ & \dots & $m+(m-2)$ & $m+(m-1)$\\
        \vdots & \vdots & \vdots & \vdots & \vdots\\
        $(n-2)m$ & $(n-2)m+1$ & \dots & $(n-2)m+(m-2)$ & $(n-2)m+(m-1)$\\
        $(n-1)m$ & $(n-1)m+1$ & \dots & $(n-1)m+(m-2)$ & $(n-1)m+(m-1)$
    \end{tabular}
\end{table}

which consists of integers from $0$ to $mn-1$, forming a complete residue system (mod $mn$).

Clearly, each row of the array forms a complete residue system (mod $m$) and all the elements of any column are congruent to each other (mod $m$). Now there are two types of columns: $\phi(m)$ columns containing only co-primes to $m$ and the rest containing none of them. (\textcolor{blue}{lähde?})

Now consider the co-prime columns. Every column forms a complete residue system (mod $n$)
%(\textcolor{blue}{LeVeque: thm. 3.5, chp. 3.2, p. 52})
\cite{LeVeque}, meaning each includes $\phi(n)$ elements co-prime to $n$. Counting $\phi(n)$ elements from all the $\phi(m)$ columns we get in total $\phi(m)\phi(n)$ numbers that are relatively prime to both $m$ and $n$.

On the other hand, since $gcd(m,n)=1$, an integer $k$ is co-prime to $mn$ if and only if both $gcd(m,k)=1$ and $gcd(n,k)=1$ are true. Hence there are $\phi(m)\phi(n)$ numbers relatively prime to $mn$. Thus by definition $\phi(mn)=\phi(m)\phi(n)$.

The case $m=1$ or $n=1$ is trivial, since $\phi(1)=1$ and hence $\phi(mn)=\phi(m)\phi(n)$.

\end{proof}

\end{theorem}

The definition of Euler's totient function is rather verbal, making it somewhat inconvenient to handle in computation. Fortunately, there is a simple formula to calculate the value as a product involving primes that divide $n$.

\begin{theorem}{\emph{Euler's product formula}}
\label{thm:product}

%\textcolor{blue}{Hardy-Wright: chapter 5.5, theorem 62, p. 64}\\
%\textcolor{red}{Oisko jossain parempi lähde}

\begin{equation*}
    \phi(n) = n \prod_{p \vert n} \left(1 - \frac{1}{p}\right)
\end{equation*}

where $\prod_{p \vert n} \left(1 - \frac{1}{p}\right)$ means the product over \emph{distinct} primes that divide $n$.

\begin{proof}

Assume first that $n = p^k$, where $p\in \mathbb{P}$. Now for every $x$, for which $gdc(p^k,x)>1$, holds $x=mp^{k-1}$ for some $m\in \{1,2,...,p^{k-1}\}$.

Hence
\begin{equation*}
    \phi(n)=\phi(p^k)=p^k-p^{k-1}=p^k-\frac{p^k}{p}=\left(1-\frac{1}{p}\right)p^k=\left(1-\frac{1}{p}\right)n.
\end{equation*}

Then, in the general case, assume $n=p_1^{k_1} p_2^{k_2} \cdot \cdot \cdot p_r^{k_r}=\prod_{i=1}^r p_i^{k_i}$, where $p_1,p_2,...,p_r$ are distinct primes that divide $n$ and $k_1,k_2,...,k_r$ their powers respectively. 

% Toi "their powers" ei oo hyvä, joku kertaluku tai muu käsite pitäis kaivaa siihen

Now, since $\phi$ is a multiplicative function
\begin{align*}
    \phi(n) & = \phi(p_1^{k_1} p_1^{k_1} \cdot \cdot \cdot p_r^{k_r})\\
    & = \phi(p_1^{k_1})\,\phi(p_2^{k_2}) \cdot \cdot \cdot \phi(p_r^{k_r})\\
    & = \left(1-\frac{1}{p_1}\right)p_1^{k_1} \left(1-\frac{1}{p_2}\right)p_2^{k_2} \cdot \cdot \cdot \left(1-\frac{1}{p_r}\right)p_r^{k_r}\\
    & = \prod_{i=1}^r \left(1-\frac{1}{p_i}\right) p_i^{k_i}\\
    & = n \prod_{p \vert n} \left(1 - \frac{1}{p}\right).
\end{align*}

\end{proof}

\end{theorem}

When it comes to primes, the value of the totient function is easy to deduce. By definition, primes are not divisible by any other number than themself and one, yielding the following lemma.

\begin{lemma}
\label{thm:phiprime}
For every $p \in \mathbb{P}$ holds $\phi(p) = p-1$\,.

\begin{proof}

Let $n\in\mathbb{P}$. Now the only prime that divides $n$ is $n$ itself. Hence by the Euler's product formula
\begin{equation*}
    \phi(n) = n \prod_{p \vert n} \left(1 - \frac{1}{p}\right) = n\left(1-\frac{1}{n}\right) = n-1\,.
\end{equation*}

\end{proof}

\end{lemma}

\section{Relevant functions and lemmas}

Building up to the order of the totient function, we must introduce few functions and theorems that are used in the proof of the lower limit. Since all of the results of this chapter serve mainly as tools, proof for many of them is not elaborated.

\subsection{Mertens' theorem}

The most important (and trickiest at the same time) of these is the Mertens' theorem, from which the mysterious $\gamma$ pops up.
\textcolor{red}{Okei jätän näiden sepustutamisen siiheksi kun on se todistus ees valmis}

\begin{theorem}
\textit{Mertens' theorem} \cite{HardyWright}
%(\textcolor{blue}{Hardy-Wright: thm. 429, chp. 22.8, p. 466})
\label{thm:mertens}

\begin{equation*}
    \lim_{n \rightarrow \infty} \log n \prod_{p\leq n} \left(1-\frac{1}{p}\right) = e^{-\gamma}
\end{equation*}

where $\gamma$ is the Euler's constant.

\begin{proof}

\textcolor{gray}{Placeholder for a sketch of the proof.}

\end{proof}

\end{theorem}

\begin{definition}{\emph{Euler-Mascheroni constant} (\textcolor{blue}{Wolfram MathWorld: Euler-Mascheroni constant} (\textcolor{red}{Jätän vielä sikseen kunnes todistus on laadittu}))}

The Euler-Mascheroni constant $\gamma$ equals the limit of the difference of the harmonic series and natural logarithm,
\begin{equation*}
    \gamma=\lim_{n\rightarrow\infty} \left(\sum_{k=1}^n \frac{1}{k} - \log n\right) \approx 0,57721566\,.
\end{equation*}
The constant appears in the Mertens' theorem and later in the lower limit of the totient function, yet more detailed consideration goes beyond the scope of this thesis.

\end{definition}

\subsection{Other relevant functions}

\begin{definition}
\textit{The sigma function} \cite{HardyWright}
%(\textcolor{blue}{Hardy-Wright: chp. 16.7, p. 310})

\begin{equation*}
    \sigma(n)=\sum_{d\vert n} d\,,
\end{equation*}

meaning the value of $\sigma(n)$ is the sum of the divisors of $n$.
\end{definition}

\begin{lemma}
\label{lemma:sigma}
Let $n=p_1^{k_1}p_2^{k_2}\cdots p_r^{k_r}$ be the prime factorization of $n$, where $p_1,p_2,...,p_r$ are distinct primes. Then
\begin{equation*}
    \sigma(n)=\prod_{i=1}^r \frac{p_i^{k_i+1}-1}{p_i-1}\,,
\end{equation*}
also denoted by
\begin{equation*}
    \sigma(n) = \prod_{p\vert n} \frac{p^{k+1}-1}{p-1}\,.
\end{equation*}

\begin{proof}
\cite{HardyWright} \textcolor{red}{Handlaa nää}
%\textcolor{blue}{Hardy-Wright: thm. 275, chp. 16.7, p. 311 }.
\end{proof}
\end{lemma}

\begin{lemma}
\label{thm:sigmafii}
\begin{equation*}
    \frac{\phi(n)\,\sigma(n)}{n^2}<1
\end{equation*}
\cite{HardyWright} \textcolor{red}{Öhh mites tää kantsis}

\begin{proof}
By the Euler's product formula and lemma \ref{lemma:sigma} we get
\begin{align*}
    \phi(n)\,\sigma(n) & = n\prod_{p\vert n}\left(1-\frac{1}{p}\right) \cdot \prod_{p\vert n} \frac{p^{k+1}-1}{p-1}\\
    & = n\prod_{p\vert n}\left(1-\frac{1}{p}\right) \cdot \prod_{p\vert n} p^k \cdot \prod_{p\vert n} \frac{p-\frac{1}{p^k}}{p-1}\\
    & = n\prod_{p\vert n}\left(1-\frac{1}{p}\right) \cdot n \prod_{p\vert n}\frac{1-\frac{1}{p^{k+1}}}{1-\frac{1}{p}}\\
    & = n^2 \prod_{p\vert n} \left( \frac{1-\frac{1}{p^{k+1}}}{1-\frac{1}{p}} - \frac{1-\frac{1}{p^{k+1}}}{p-1}\right)\\
    & = n^2 \prod_{p\vert n} \frac{p-1-\frac{1}{p^k}+\frac{1}{p^{k+1}}}{p-1}\\
    & = n^2 \prod_{p\vert n} \frac{p-1-(p-1)\frac{1}{p^{k+1}})}{p-1}\\
    & = n^2 \prod_{p\vert n} \left(1-\frac{1}{p^{k+1}}\right) < n^2\,.
\end{align*}
Equivivalently
\begin{equation*}
    \frac{\phi(n)\,\sigma(n)}{n^2}<1\,.
\end{equation*}
\end{proof}
\end{lemma}


\begin{definition}
\textit{Chebyshev function} \cite{HardyWright}
%(\textcolor{blue}{Hardy-Wright: chp. 22.1, p. 451})

\begin{equation*}
    \vartheta(x)=\sum_{p\leq x} \log p = \log \prod_{p\leq x} p\,,
\end{equation*}
where $x\in\mathbb{R}$ and $p\in\mathbb{P}$.

\end{definition}

\begin{lemma}
\label{lemma:vartheta}
For the function $\vartheta(x)$ holds
\begin{equation*}
    \vartheta(x) < Ax\,,
\end{equation*}
where $x\geq2\in\mathbb{R}$, $A$ is a real constant.

\begin{proof}
\cite{HardyWright} \textcolor{red}{Handlaa nää}
%\textcolor{blue}{Hardy-Wright: thm. 414, chp. 22.2, p. 453}.
\end{proof}
\end{lemma}

\begin{definition}
\textit{Riemann zeta function} \cite{HardyWright}
%(\textcolor{blue}{Hardy-Wright: chp. 17.2, p. 320})

\begin{equation*}
    \zeta(s)=\sum_{n=1}^\infty \frac{1}{n^s}\,,
\end{equation*}
where $s\in\mathbb{R}$.
\end{definition}

\begin{theorem}
\textit{Limit of zeta function}
\label{thm:zeta}
\begin{equation}
    \lim_{s\rightarrow\infty} \zeta(s) = 1\,.
\end{equation}
\begin{proof}
\textcolor{red}{ETSI JOSTAIN}
\end{proof}
\end{theorem}

\begin{lemma}
\label{lemma:zeta}
For all $s>1\in\mathbb{R}$, 
\begin{equation*}
    \zeta(s)=\prod_p \frac{1}{1-\frac{1}{p^s}}\,.
\end{equation*}
\begin{proof}
\cite{HardyWright} \textcolor{red}{Handlaa nää}
%\textcolor{blue}{Hardy-Wright: thm. 280, chp. 17.2, p. 320}.
\end{proof}
\end{lemma}

\section{The limits of Euler's totient function}

Finally reaching the point, in which we are equipped to start getting into the order of the totient function, let us still consider, why it is interesting at all. As pondered before, the Euler's product formula gives the totient function a more computable form. However, using it requires factorization of $n$, which still makes it difficult to estimate the size of $\phi(n)$ as $n$ gets bigger.

Let us amuse ourselves with an example \cite{Pomerance}: let $n = 2^p - 1 \in \mathbb{P}$ be so called Mersenne prime, meaning also $p \in \mathbb{P}$.

By theorem \ref{thm:phiprime} we know $\phi(n) = n - 1$. On the other hand, from Euler's product formula follows that $\phi(n+1) = \phi(2^p) = 2^p(1-\frac{1}{2}) = \frac{2^p}{2} = \frac{n+1}{2}$.

Now we see that while $n$ and $n+1$ differ from each other only insignificantly, $\phi(n+1)$ is half the size of $\phi(n)$.

All this said, next we will prove the exact limits of the totient function, starting with the fairly obvious upper limit and then diving into a detailed proof of the lower limit.

\subsection{Upper limit of Euler's totient function}

The maximum value of $\phi(n)$ given $n$ is easy to define with theorem \ref{thm:phiprime}.

\begin{theorem}
\textit{Upper limit of the totient function} \cite{HardyWright}
%(\textcolor{blue}{Hardy-Wright: thm. 326, chp. 18.4, p. 352})

For every $n \geq 2$ holds $\phi(n) \leq n-1$ and
\begin{equation*}
    \limsup{\frac{\phi(n)}{n}} = 1\,.
\end{equation*}

\begin{proof}

By definition, $\phi(n) \leq n$ beacause there are $n$ elements in the set $\{1,2,...,n\}$. Also, for every $n \geq 2$ holds $gcd(n,n) = n \neq 1$. Thus, $\phi(n) \leq n-1$.

On the other hand, according to theorem \ref{thm:phiprime}, $\phi(p) = p-1$ for every $p\in\mathbb{P}$.
% TÄÄ ON VÄÄRIN SANOA NÄIN
% Because there are infinitely many primes, this means that $n-1$ is, in fact, the limit superior of Euler's totient function.
Now, because there are infinitely many primes (\textcolor{blue}{lähde?}),
\begin{equation*}
    \limsup{\frac{\phi(n)}{n}} = \lim \frac{n-1}{n} = 1\,.
\end{equation*}

\end{proof}

\end{theorem}

\subsection{Lower limit of Euler's totient function}

How small $\phi(n)$ can be as $n$ grows, is much less trivial a question to answer. However, it can be shown that the value of $\phi(n)$ is proportional to $\frac{n}{\log\log n}$. The rest of this paper will cover the proof of the exact limit inferior of the totient function, following the proof of theorem $328$
%from \textit{Hardy \& Wright: Introduction to the Theory of Numbers},
in \cite{HardyWright} chapter 22.9, p. 467. \textcolor{red}{Myös tässä nyt joku järki viittaamiseen}

\begin{theorem}
\textit{Lower limit of the totient function} \cite{HardyWright}
%(\textcolor{blue}{Hardy-Wright: thm. 328, chp. 18.4, p. 352})

\begin{equation*}
    \liminf{\frac{\phi(n)\,\log\log n}{n}}=e^{-\gamma}\,,
\end{equation*}
where $\gamma$ is the Euler's constant.

\begin{proof}

Let us prove the claim by showing $\liminf{f(n)} = 1$, when
\begin{equation*}
    f(n)= \frac{\phi(n)\,e^\gamma \log\log n}{n}\,,
\end{equation*}
and $\gamma$ is the Euler's constant.

The proof is based on finding two functions $F_1(t)$ and $F_2(t)$, the limits of which are both $\lim_{t\rightarrow \infty} F_1(t) = 1$ and $\lim_{t\rightarrow \infty} F_2(t) = 1$. First we show that 
\begin{equation}
\label{eq:first}
    f(n) \geq F_1(\log n)\text{ for all }n\geq 3
\end{equation}
and in the second part that
\begin{equation}
\label{eq:second}
    f(n_j) \leq \frac{1}{F_2(j)}\text{ for some infinite increasing sequence }n_2, n_3,...
\end{equation}

Let $p_1,p_2,...,p_{r-\rho} \leq \log n$ and $p_{r-\rho+1},...,p_r > \log n$ be the prime factors of $n$. In other words, the number $n$ has $r$ prime factors, $\rho$ of which are greater than $\log n$.

Now
\begin{equation*}
    (\log n)^\rho < p_{r-\rho+1} \cdot p_{r-\rho+2} \cdots p_r \leq n\,,
\end{equation*}
which yields
\begin{equation*}
    \rho < \frac{\log n}{\log\log n}\,.
\end{equation*}

Thus, there are less than $\frac{\log n}{\log\log n}$ prime factors greater than $\log n$.

By the Euler's product formula (theorem \ref{thm:product})
\begin{align*}
    \frac{\phi(n)}{n} & = \prod_{i=1}^r\left(1-\frac{1}{p_i}\right)\\
    & = \prod_{i=1}^{r-\rho}\left(1-\frac{1}{p_i}\right) \prod_{i=r-\rho+1}^r\left(1-\frac{1}{p_i}\right)\\
    & = \prod_{p\leq\log n}\left(1-\frac{1}{p}\right) \prod_{p>\log n}\left(1-\frac{1}{p}\right)\\
    & \geq \left(1-\frac{1}{\log n}\right)^\rho \prod_{p\leq\log n}\left(1-\frac{1}{p}\right) \\
    & > \left(1-\frac{1}{\log n}\right)^\frac{\log n}{\log \log n} \prod_{p\leq\log n}\left(1-\frac{1}{p}\right)\,.
\end{align*}

Hence, we can define
\begin{equation*}
    F_1(t)=e^\gamma \log t \left(1-\frac{1}{t}\right)^\frac{t}{\log t} \prod_{p\leq t} \left(1-\frac{1}{p}\right)\,,
\end{equation*}
because by the inequality above
\begin{align*}
    F_1(\log n) & = e^\gamma \log \log n \left(1-\frac{1}{\log n}\right)^\frac{\log n}{\log \log n} \prod_{p\leq \log n} \left(1-\frac{1}{p}\right)\\
    & \leq \frac{\phi(n)}{n} e^\gamma \log\log n = f(n)
\end{align*}
and by the Mertens' theorem (theorem \ref{thm:mertens})
\begin{align*}
    \lim_{t \rightarrow \infty} F_1(t) & = \lim_{t \rightarrow \infty} e^\gamma \log t \left(1-\frac{1}{t}\right)^\frac{t}{\log t} \prod_{p\leq t} \left(1-\frac{1}{p}\right)\\
    & = \lim_{t \rightarrow \infty} e^\gamma \left( 1-\frac{1}{t}\right)^\frac{t}{\log t} \left(\log t \prod_{p\leq t} \left(1-\frac{1}{p}\right) \right)\\
    & = \lim_{t \rightarrow \infty} e^\gamma \left( 1-\frac{1}{t}\right)^\frac{t}{\log t} e^{-\gamma}\\
    & = \lim_{t \rightarrow \infty} \left( 1-\frac{1}{t}\right)^\frac{t}{\log t}\\
    & = 1\,.
\end{align*}

Now we have proved the part (\ref{eq:first}) and showed that
$\liminf{f(n)}\geq 1$.

Next, to prove the part (\ref{eq:second}), let us define
\begin{equation*}
    g(n)=\frac{\sigma(n)}{n\,e^\gamma \log\log n}
\end{equation*}
and show that $g(n_j) \geq F_2(j)$ for an infinite increasing sequence $n_2,n_3\dots$. The desired result will follow from theorem \ref{thm:sigmafii}.

Let
\begin{equation*}
    n_j=\prod_{p\leq e^j} p^j\text{, where } j\geq 2\,.
\end{equation*}

By the lemma \ref{lemma:vartheta}
\begin{equation*}
    \log n_j = \log \prod_{p\leq e^j} p^j = j \log \prod_{p\leq e^j} p = j\,\vartheta(e^j) \leq A\,j\,e^j\,,
\end{equation*}
where $A$ is a real constant.

Hence
\begin{equation*}
\label{eq:lognj}
    \log \log n_j = \log Aje^j = \log A + \log j + \log e^j = \log A + \log j + j\,.
\end{equation*}

Since $n_j$ is the product of all primes smaller than $e^j$ to the power of $j$, by the lemma \ref{lemma:sigma} we have
\begin{equation*}
    \sigma(n_j) = \prod_{p\leq e^j} \frac{p^{j+1}-1}{p-1}
\end{equation*}
and
\begin{equation*}
    \frac{\sigma(n_j)}{n_j} = \prod_{p\leq e^j} \frac{p^{j+1}-1}{(p-1)p^j} = \prod_{p\leq e^j} \frac{p^{j+1}\left(1-\frac{1}{p^{j+1}}\right)}{p^{j+1}\left(1-\frac{1}{p}\right)} = \prod_{p\leq e^j} \frac{1-\frac{1}{p^{j+1}}}{1-\frac{1}{p}}\,.
\end{equation*}

Also, by the lemma \ref{lemma:zeta}
\begin{equation*}
    \prod_{p\leq e^j}\left(1-\frac{1}{p^{j+1}}\right) > \prod \left(1-\frac{1}{p^{j+1}}\right) = \frac{1}{\zeta(j+1)}\,.
\end{equation*}

Now we can define
\begin{equation*}
    F_2(t)=\frac{1}{e^\gamma\,\zeta(t+1)(B+t+\log t)} \prod_{p\leq e^t} \left(\frac{1}{1-\frac{1}{p}}\right)\,,
\end{equation*}
where $B=\log A$ is a suitable real constant.

This is, by combining the results above
\begin{align*}
    F_2(j)& = \frac{1}{e^\gamma\,\zeta(j+1)(B+j+\log j)} \prod_{p\leq e^j} \left(\frac{1}{1-\frac{1}{p}}\right)\\
    & \leq \frac{1}{e^\gamma\,\log \log n_j} \prod_{p\leq e^j} \frac{1-\frac{1}{p^{j+1}}}{1-\frac{1}{p}}\\
    & = \frac{\sigma(n_j)}{n_j\,e^\gamma\,\log \log n_j} = g(n_j)\,.
\end{align*}

By the Mertens' third theorem (theorem \ref{thm:mertens})

\begin{equation*}
    \lim_{t \rightarrow \infty} \prod_{p\leq e^t} \left(\frac{1}{1-\frac{1}{p}}\right) = \lim_{t \rightarrow \infty} \frac{1}{\prod_{p\leq e^t} \left(1-\frac{1}{p}\right)} = \left(\frac{e^{-\gamma}}{\log e^t}\right)^{-1} = e^\gamma\,t
\end{equation*}

and hence, keeping theorem \ref{thm:zeta} in mind
\begin{align*}
    \lim_{t \rightarrow \infty} F_2(t) & = \lim_{t \rightarrow \infty} \frac{1}{e^\gamma\,\zeta(t+1)(B+t+\log t)} \prod_{p\leq e^t} \left(\frac{1}{1-\frac{1}{p}}\right)\\
    & = \lim_{t \rightarrow \infty} \frac{e^\gamma\,t}{e^\gamma\,\zeta(t+1)(B+t+\log t)}\\
    & = \lim_{t \rightarrow \infty} \frac{t}{\zeta(t+1)(B+t+\log t)}\\
    & = \lim_{t \rightarrow \infty} \frac{t}{B+t+\log t}\\
    & = 1\,.
\end{align*}

By the theorem \ref{thm:sigmafii}
\begin{equation*}
    f(n)\,g(n) = \frac{\phi(n)\,e^\gamma \log\log n}{n} \cdot \frac{\sigma(n)}{n\,e^\gamma \log\log n} = \frac{\phi(n)\sigma(n)}{n^2}<1
\end{equation*}

and since $g(n_j) \geq F_2(j)$

\begin{equation*}
    f(n_j)\leq \frac{1}{F_2(j)}\,.
\end{equation*}

Thus we have proved the part (\ref{eq:second}) and showed that $\liminf{f(n)\leq 1}$.

Altogether, from the parts (\ref{eq:first}) and (\ref{eq:second}), we get that the limit inferior of $f(n)$ must be
\begin{equation*}
    \liminf{\frac{\phi(n)\,e^\gamma \log\log n}{n}}=\liminf{f(n)}=1\,
\end{equation*}
and equivalently
\begin{equation*}
    \liminf{\frac{\phi(n)\,\log\log n}{n}}=e^{-\gamma}\,.
\end{equation*}

\end{proof}

\end{theorem}

\nocite{*}
\printbibliography

\end{document}