\documentclass{article}
\usepackage[utf8]{inputenc}
\usepackage{lmodern}
\usepackage[T1]{fontenc}
\usepackage[english]{babel}
\usepackage{amsmath}
\usepackage{amsthm}
\usepackage{amssymb}

\usepackage[backend=biber,style=numeric]{biblatex}
\usepackage{csquotes}
\addbibresource{viitteet.bib}

\title{The size of Euler's totient function when $n \rightarrow \infty$}
\author{Elli Kiiski}
\date{2021}

\theoremstyle{definition}
\newtheorem{definition}[subsection]{Definition}
\newtheorem{notation}[subsection]{Notation}
\newtheorem{theorem}[subsection]{Theorem}

\begin{document}

\maketitle

\newpage

\section{Summary}

Viittaus \cite{HardyWright} toinenkin \cite{Saksman}

\section{Introduction}

\section{Notation and definitons}

\begin{notation}{\emph{Divisibility $a \vert b$}}

Let $a \in \mathbb{Z}$ and $b \in \mathbb{Z}$ be such that $b$ is divisible by $a$. This is denoted by $a \vert b$.

\end{notation}

\begin{definition}{\emph{Greatest common divisor, $gcd(a, b)$}}

Let $a \not= 0$ and $b \not= 0$. There is a unique $d \in \mathbb{N}$ with following properties:

\begin{enumerate}
 \item $d \vert a$ and $d \vert b$
 \item if $d' \vert a$ and $d' \vert b$, then $d' \vert d$
\end{enumerate}

The number $d$ is called the greatest common divisor of $a$ and $b$. It's denoted by $gcd(a,b) = d$.

\end{definition}

\begin{definition}{\emph{Prime number}}

Integer $p\in\mathbb{N}$ is a prime, if $p \geq 2$ and for every $k\in\mathbb{N}$ holds that if $k \vert p$ then $k\in{1, p}$. The set of prime numbers is denoted by $\mathbb{P}$.

In other words, all integers greater than than $1$, which are only divisible by themself and $1$, are primes.

%Toisin sanoen alkulukuja ovat kaikki lukua $1$ suuremmat luonnolliset luvut, jotka ovat jaollisia vain itsellään ja luvulla $1$.

\end{definition}

\begin{definition}{\emph{Co-primes}}

If $gcd(a,b) = 1$, $a$ and $b$ are called co-primes or relative primes.

\end{definition}

\section{Euler's totient function}

%Eulerin $\phi$-funktio on lukuteoreettinen funktio, eli se kuvautuu luonnollisilta luvuilta luonnollisille luvuille. Nönnönnöö

\begin{definition}{\emph{Eulerin totient function $\phi: \mathbb{N} \rightarrow \mathbb{N}$}}

It's set that $\phi(1) = 1$. For all $n \geq 2$, $\phi(n)$ is the number of integers $a \in \{1,2,...,n\}$, for which $gcd(a,n) = 1$.

AIKA SURKEE SELITYS
That is, the value of the totient function in $n \in \mathbb{N}$ is the number of natural numbers smaller than $n$, which are its co-primes.

%Toisin sanoen Eulerin $\phi$-funktion arvo luonnollisella luvulla $n$ on sitä pienempien luonnolisten lukujen määrä, jotka ovat alkulukuja sen suhteen.

\end{definition}

\newpage

\begin{theorem}{Euler's product formula}

\begin{equation*}
    \phi(n) = n \prod_{p \vert n} (1 - \frac{1}{p})
\end{equation*}

where $\prod_{p \vert n} (1 - \frac{1}{p})$ means the product over \emph{distinct} primes that divide $n$.

\begin{proof}

KIKKI

\end{proof}

\end{theorem}

\section{The bounds of Euler's totient function JÄIN TÄHÄN}

Kuten pätee monille lukuteoreettisille funktioille, myös Eulerin $\phi$-funktion arvo heittelehtii $n$:n kasvaessa OIKEI LUOVUTAN, VAIHDAN ENKKUUN

\subsection{Eulerin $\phi$-funktion yläraja}

\begin{theorem}{\emph{Eulerin $\phi$-funktion yläraja}}

Kaikilla luonnollisilla luvuilla $n \geq 2$ pätee $\phi(n) < n$.

\begin{proof}

Suoraan määritelmästä seuraa, että $\phi(n) \leq n$, koska joukossa $\{1,2,...,n\}$ on $n$ alkiota ja siten niiden joukosta ei voi löytyä yli $n$ kappaletta ehtoa täyttävää lukua. Lisäksi jokaisella $n$ pätee $syt(n,n) = n$. Täten millään $n \geq 2$ ei voi olla $\phi(n) = n$.

Siis $\phi(n) < n$ jokaisella $n \geq 2$.

\end{proof}

\end{theorem}

\begin{theorem}{\emph{Alkuluvuilla $\phi(p)=p-1$}}

Jokaisella alkuluvulla $p\in\mathbb{P}$ pätee $\phi(p)=p-1$.

\begin{proof}

Olkoon $p\in\mathbb{P}$. Tällöin jokaisella $k<p$, $k\in\mathbb{N}$ pätee $syt(k, p)=1$, mistä seuraa suoraan $\phi(p)=p-1$. PITÄISIKÖ TÄÄ TODISTAA PAREMMIN

\end{proof}

\end{theorem}

\begin{theorem}{$\phi$-funktion pienin yläraja}

Jokaisella $n \in \mathbb{N}$ pätee $\phi(n)\leq {n-1}$.

\begin{proof}

Tulos saadaan suoraan yhdistämällä lauseet ÄSKEINEN ja SITÄ EDELLINEN.

\end{proof}

\end{theorem}

\subsection{Eulerin $\phi$-funktion alaraja}

\subsection{$\phi(n) < \sqrt(n)$?}

Lähdetään tutkimaan $\phi$-funktion alarajaa tarkastelemalla onko olemassa suuria luonnollisia lukuja, joilla $\phi(n) < \sqrt{n}$. Huomataan, että ainakin vielä luvulla $n=6$ pätee $\phi(6)=2<\sqrt(6)$, mutta sen jälkeen arvot näyttäisivät järjestään ylittävän vastaavan neliöjuuren arvon.

Tarkastellaan tilannetta tarkemmin jos osataan ehehe

\section{Eulerin $\phi$-funktion keskiarvo}

\section{Asiaaa}

\section{Lähteet}

\printbibliography[heading=none]

\end{document}