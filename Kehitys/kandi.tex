\documentclass{article}
\usepackage[utf8]{inputenc}
\usepackage{lmodern}
\usepackage[T1]{fontenc}
\usepackage[finnish]{babel}
\usepackage{amsmath}
\usepackage{amsthm}
\usepackage{amssymb}

\usepackage[backend=biber,style=numeric]{biblatex}
\usepackage{csquotes}
\addbibresource{viitteet.bib}

\title{Eulerin $\phi$-funktion koko kun $n \rightarrow \infty$}
\author{Elli Kiiski}
\date{2020}

\theoremstyle{definition}
\newtheorem{maaritelma}[subsection]{Määritelmä}
\newtheorem{merkinta}[subsection]{Merkintätapa}
\newtheorem{lause}[subsection]{Lause}

\begin{document}

\maketitle

\newpage

\section{Tiivistelmä}

Viittaus \cite{HardyWright} toinenkin \cite{Saksman}

\section{Johdanto}

\section{Määritelmiä ja merkintätapoja}

\begin{merkinta}{\emph{Jaollisuus $a \vert b$}}

Olkoot $a \in \mathbb{Z}$ ja $b \in \mathbb{Z}$ siten, että luku $b$ on jaollinen luvulla $a$. Tällöin merkitään $a \vert b$.

\end{merkinta}

\begin{maaritelma}{\emph{Suurin yhteinen tekijä, $syt(a, b)$}}

Olkoot $a \not= 0$ ja $b \not= 0$. Tällöin on olemassa yksiselitteinen $d \in \mathbb{N}$, jolla on seuraavat ominaisuudet:

\begin{enumerate}
 \item $d \vert a$ ja $d \vert b$
 \item jos $d' \vert a$ ja $d' \vert b$, niin $d' \vert d$
\end{enumerate}

Lukua $d$ kutsutaan lukujen $a$ ja $b$ suurimmaksi yhteiseksi tekijäksi, ja merkitään $syt(a,b) = d$.

\end{maaritelma}

\begin{maaritelma}{\emph{Alkuluku}}

Luku $p\in\mathbb{N}$ on alkuluku, jos $p \geq 2$ ja jokaisella $k\in\mathbb{N}$ jaollisuudesta $k \vert p$ seuraa $k\in{1, p}$. Tällöin merkitään $p\in\mathbb{P}$.

Toisin sanoen alkulukuja ovat kaikki lukua $1$ suuremmat luonnolliset luvut, jotka ovat jaollisia vain itsellään ja luvulla $1$.

\end{maaritelma}

\begin{maaritelma}{\emph{Suhteellinen alkuluku}}

Jos $syt(a,b) = 1$, kutsutaan lukuja $a$ ja $b$ suhteellisiksi alkuluvuiksi tai alkuluvuiksi toistensa suhteen.

\end{maaritelma}

\section{Eulerin $\phi$-funktio ja Möbiuksen $\mu$-funktio}

Määritellään seuraavaksi kaksi funktiota: itse $\phi$-funktio sekä myöhemmin hyödylliseksi apufunktioksi osoittautuva $\mu$-funktio.

\subsection{Eulerin $\phi$-funktio}

Eulerin $\phi$-funktio on lukuteoreettinen funktio, eli se kuvautuu luonnollisilta luvuilta luonnollisille luvuille. Nönnönnöö

\begin{maaritelma}{\emph{Eulerin $\phi$-funktio $\phi: \mathbb{N} \rightarrow \mathbb{N}$}}

Määritetään $\phi(1) = 1$. Kaikilla $n \geq 2$, $\phi(n)$ on lukujen $a \in \{1,2,...,n\}$ määrä, joille pätee $syt(a,n) = 1$.

Toisin sanoen Eulerin $\phi$-funktion arvo luonnollisella luvulla $n$ on sitä pienempien luonnolisten lukujen määrä, jotka ovat alkulukuja sen suhteen.

\end{maaritelma}

\subsection{Möbiuksen $\mu$-funktio}

Möbiuksen $\mu$-funktio puolestaan nönnönöö

\begin{maaritelma}{\emph{Möbiuksen $\mu$-funktio $\mu: \mathbb{N} \rightarrow \mathbb{N}$}}

Määritelmä tähän

\end{maaritelma}

\section{Eulerin $\phi$-funktion rajat}

Todisteteaan funktion triviaali yläraja ja tutkitaan mutkikkaampaa alarajaa.

\subsection{Eulerin $\phi$-funktion yläraja}

\begin{lause}{\emph{Eulerin $\phi$-funktion yläraja}}

Kaikilla luonnollisilla luvuilla $n \geq 2$ pätee $\phi(n) < n$.

\begin{proof}

Suoraan määritelmästä seuraa, että $\phi(n) \leq n$, koska joukossa $\{1,2,...,n\}$ on $n$ alkiota ja siten niiden joukosta ei voi löytyä yli $n$ kappaletta ehtoa täyttävää lukua. Lisäksi jokaisella $n$ pätee $syt(n,n) = n$. Täten millään $n \geq 2$ ei voi olla $\phi(n) = n$.

Siis $\phi(n) < n$ jokaisella $n \geq 2$.

\end{proof}

\end{lause}

\begin{lause}{\emph{Alkuluvuilla $\phi(p)=p-1$}}

Jokaisella alkuluvulla $p\in\mathbb{P}$ pätee $\phi(p)=p-1$.

\begin{proof}

Olkoon $p\in\mathbb{P}$. Tällöin jokaisella $k<p$, $k\in\mathbb{N}$ pätee $syt(k, p)=1$, mistä seuraa suoraan $\phi(p)=p-1$. PITÄISIKÖ TÄÄ TODISTAA PAREMMIN

\end{proof}

\end{lause}

\begin{lause}{$\phi$-funktion pienin yläraja}

Jokaisella $n \in \mathbb{N}$ pätee $\phi(n)\leq {n-1}$.

\begin{proof}

Tulos saadaan suoraan yhdistämällä lauseet ÄSKEINEN ja SITÄ EDELLINEN.

\end{proof}

\end{lause}

\subsection{Eulerin $\phi$-funktion alaraja}

\subsection{$\phi(n) < \sqrt(n)$?}

Lähdetään tutkimaan $\phi$-funktion alarajaa tarkastelemalla onko olemassa suuria luonnollisia lukuja, joilla $\phi(n) < \sqrt{n}$. Huomataan, että ainakin vielä luvulla $n=6$ pätee $\phi(6)=2<\sqrt(6)$, mutta sen jälkeen arvot näyttäisivät järjestään ylittävän vastaavan neliöjuuren arvon.

Tarkastellaan tilannetta tarkemmin jos osataan ehehe

\section{Asiaaa}

\section{Asiaaaa}

\section{Lähteet}

\printbibliography[heading=none]

\end{document}