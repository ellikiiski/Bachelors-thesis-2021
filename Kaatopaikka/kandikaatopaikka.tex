\documentclass{article}
\usepackage[utf8]{inputenc}
\usepackage{parskip}
\usepackage{amsmath}
\usepackage{amssymb}
\usepackage{amsthm}

\theoremstyle{definition}
\newtheorem{definition}[subsection]{Definition}
\newtheorem{notation}[subsection]{Notation}
\newtheorem{agreement}[subsection]{Agreement}
\newtheorem{theorem}[subsection]{Theorem}

\title{Kandikaatopaikka}
\author{Elli Kiiski}
\date{\today}

\begin{document}
{\large
Elli Kiiski
\par
\textbf{2021 Kandikaatopaikka}
}
\vspace{0.5cm}

\section{Hardy-Wrightin todistuksen perkaamista}

\textit{G. H. Hardyn} ja \textit{E. M. Wrightin} kirjan \textit{An Introduction to the theory of numbers} sivulla 469 olevan $\phi$-funktion alarajan todistuksen läpikäyntiä.

\subsection{Määrittely: mitä todistetaan}

Aloitetaan määrittelemällä kuvaus
\begin{equation*}
    f(n)= \frac{\phi(n)e^\gamma \log\log n}{n},
\end{equation*}
missä $\gamma$ on Eulerin vakio.

% Mikä on Eulerin vakio ja mistä se hyppää tähän
% Mistä myöskään e yhtäkkiä hyppää tähän

Halutaan todistaa, että $\liminf f(n)=1$, mikä on yhtäpitävää sen kanssa, että $\phi$-funktion alaraja on $\frac{n}{e^\gamma \log\log n}$.

\subsection{Määrittely: miten todistetaan}

Riittää löytää funktiot $F_1(t)$ ja $F_2(t)$, joille pätee
\begin{enumerate}
\label{ehdot}
    \item $\lim_{t\rightarrow \infty} F_1(t) = 1$ ja $\lim_{t\rightarrow \infty} F_2(t) = 1$
    \item $f(n) \geq F_1(\log n)$ kaikilla $n\geq 3$
    \item $f(n_j) \leq \frac{1}{F_2(j)}$ äärettömällä kasvavalla jonolla $n_2, n_3,...$
\end{enumerate}

''Tämä tarkoittaa, että on löydetty funktio $F_1(\log n)$, jonka on sama limes infimum on yksi, mutta funktio on kaikkialla suurempi kuin $f(n)$. Tällöin funktion $f(n)$ limes infimum on enintään yksi. Vastaavasti alapuolen kanssa.''
% Ilmiselvästi täytyy selittää tuo vielä jotenkin paremmin

\subsection{Todistus osa 1: $f(n) \geq F_1(\log n)$}

Olkoot $p_1,p_2,...,p_{r-\rho} \leq \log n$ ja $p_{r-\rho+1},...,p_r > \log n$ luvun $n$ alkutekijöitä. Siis luvulla $n$ on yhteensä $r$ alkutekijää, joista $\log n$:ää suurempia on $\rho$ kappaletta.

Nyt
\begin{equation}
    (\log n)^\rho < p_{r-\rho+1} \cdot p_{r-\rho+2} \cdot \cdot \cdot p_r \leq n,
\end{equation}
mistä seuraa
\begin{equation}
    \rho < \frac{\log n}{\log\log n}.
\end{equation}

Eli $\log n$:ää suurempia alkulukutekijöitä on alle $\frac{\log n}{\log\log n}$ kappaletta.
Nyt tulokaavaa käyttäen $\phi$-funktion suhde $n$:ään voidaan ilmaista seuraavasti
\begin{align}
    \frac{\phi(n)}{n} & = \prod_{i=1}^r(1-\frac{1}{p_i})\\
    & = \prod_{i=1}^{r-\rho}(1-\frac{1}{p_i}) \prod_{i=r-\rho+1}^r(1-\frac{1}{p_i})\\
    & \geq \left(\prod_{i=1}^{r-\rho}(1-\frac{1}{p_i})\right) (1-\frac{1}{\log n})^\rho\\
    & > \left(\prod_{i=1}^{r-\rho}(1-\frac{1}{p_i})\right) (1-\frac{1}{\log n})^\frac{\log n}{\log \log n}.
\end{align}

Näin ollen voidaan valita
\begin{equation*}
    F_1(t)=e^\gamma \log t \left(1-\frac{1}{t}\right)^\frac{t}{\log t} \prod_{p\leq t} \left(1-\frac{1}{p}\right),
\end{equation*}
jolloin
\begin{align*}
    F_1(\log n) & = e^\gamma \log \log n \left(1-\frac{1}{\log n}\right)^\frac{\log n}{\log \log n} \prod_{p\leq \log n} \left(1-\frac{1}{p}\right)\\
    & = e^\gamma \log \log n \left(1-\frac{1}{\log n}\right)^\frac{\log n}{\log \log n} \prod_{i=1}^{r-\rho} \left(1-\frac{1}{p}\right)\\
    & \leq \frac{\phi(n)}{n} e^\gamma \log\log n = f(n).
\end{align*}

Kuitenkin funktiolle $F_1$ pätee Mertenin kolmannen lauseen nojalla
\begin{align*}
    \lim_{t \rightarrow \infty} F_1(t) & = \lim_{t \rightarrow \infty} e^\gamma \log t \left(1-\frac{1}{t}\right)^\frac{t}{\log t} \prod_{p\leq t} \left(1-\frac{1}{p}\right)\\
    & = \lim_{t \rightarrow \infty} e^\gamma \left( 1-\frac{1}{t}\right)^\frac{t}{\log t} \left(\log t \prod_{p\leq t} \left(1-\frac{1}{p}\right) \right)\\
    & = \lim_{t \rightarrow \infty} e^\gamma \left( 1-\frac{1}{t}\right)^\frac{t}{\log t} e^{-\gamma}\\
    & = \lim_{t \rightarrow \infty} \left( 1-\frac{1}{t}\right)^\frac{t}{\log t}\\
    & = 1
\end{align*}

Täten funktion $f$ limes infium on korkeintaan 1.

\subsection{Todistus osa 2: $f(n_j) \leq \frac{1}{F_2(j)}$}

% Tämä ei lienekään ihan niin iisi keissi

Olkoon $n_j=\prod_{p\leq e^j} p^j$, missä $j\geq 2 \in \mathbb{N}$ (?), jolloin maailman hämmentävimmän lauseen (Hardy-Wright thm 414) mukaan (*)
\begin{equation}
    \log n_j = \log \prod_{p\leq e^j} p^j = j \log \prod_{p\leq e^j} p = j\vartheta(e^j) \stackrel{*}{\leq} Aje^j
\end{equation}

ja edelleen
\begin{equation}
    \log \log n_j = \log Aje^j = \log A + \log j + \log e^j = \log A + \log j + j
\end{equation}

Jatketaan tästä pian

\section{Tulokaavan todistus}

Eulerin tulokaava arvon $\phi(n)$ laskemiseksi on hyvinkin tärkeä palanen eli todistetaan se nyt suoraan englanniksi niin ei tarvitse erikseen kääntää.

\subsection{Eulers's product formula}

\begin{theorem}{\emph{Euler's product formula}}

\begin{equation*}
    \phi(n) = n \prod_{p \vert n} \left(1 - \frac{1}{p}\right)
\end{equation*}

where $\prod_{p \vert n} (1 - \frac{1}{p})$ means the product over \emph{distinct} primes that divide $n$.

\begin{proof}

Assume first that $n = p^k$, where $p\in \mathbb{P}$. Now for every $x$, for which $gdc(p^k,x)>1$, holds $x=mp^{k-1}$ for some $m\in \{1,2,...,p^{k-1}\}$.

% Siis lukujen x, joille gcd(n,x) määrä on p^{k-1}

Hence
\begin{equation*}
    \phi(n)=\phi(p^k)=p^k-p^{k-1}=p^k-\frac{p^k}{p}=\left(1-\frac{1}{p}\right)p^k=\left(1-\frac{1}{p}\right)n.
\end{equation*}

Then, in the general case, assume $n=p_1^{k_1} p_2^{k_2} \cdot \cdot \cdot p_r^{k_r}=\prod_{i=1}^r p_i^{k_i}$, where $p_1,p_2,...,p_r$ are distinct primes that divide $n$ and $k_1,k_2,...,k_r$ their powers respectively. 

% Toi "their powers" ei oo hyvä, joku kertaluku tai muu käsite pitäis kaivaa siihen

Now, since $\phi$ is a multiplicative function
\begin{align*}
    \phi(n) & = \phi(p_1^{k_1} p_1^{k_1} \cdot \cdot \cdot p_r^{k_r})\\
    & = \phi(p_1^{k_1})\,\phi(p_2^{k_2}) \cdot \cdot \cdot \phi(p_r^{k_r})\\
    & = \left(1-\frac{1}{p_1}\right)p_1^{k_1} \left(1-\frac{1}{p_2}\right)p_2^{k_2} \cdot \cdot \cdot \left(1-\frac{1}{p_r}\right)p_r^{k_r}\\
    & = \prod_{i=1}^r \left(1-\frac{1}{p_i}\right) p_i^{k_i}\\
    & = n \prod_{p \vert n} \left(1 - \frac{1}{p}\right).
\end{align*}

\end{proof}

\end{theorem}

\section{The zeta-function}

\begin{definition}{The zeta-function}

\begin{equation*}
    \zeta(s) = \sum_{n=1]^\infty \frac{1}{n^s}}
\end{equation*}

The zeta-funtion converges, when $s>1$.

\begin{theorem}
For all $s>1$
\begin{equation*}
    \zeta(s) = \prod_p \frac{1}{1-\frac{1}{p^s}}
\end{equation*}
\end{theorem}

\end{definition}

\section{Merten's (third) theorem}

\begin{theorem}{\emph{Merten's (third) theorem}}

\begin{equation*}
    \lim_{n \rightarrow \infty} \log n \prod_{p\leq n} \left(1-\frac{1}{p}\right) = e^{-\gamma}
\end{equation*}

where $\gamma$ is the Euler's constant.
% Eulerin vakio pitänee lisätä johki, tai no katotaan miten se tossa todistuksessa tulee esiin

\begin{proof}

Oh, this seems like a työmaa

\end{proof}

\end{theorem}

\section{Edellisestä versiosta poistettua paskaa}

\subsubsection{Are there such integers $n$ that $\phi(n) < \sqrt{n}$?}

Let's begin with $\sqrt{n}$. Is there such large number $n$ that $\phi(n) < \sqrt{n}$? When checking the values of $\phi(n)$ for smaller $n$, we see that at least with $n=6$ the statement is true, as $\phi(6)=2<\sqrt{6}$. After that, however, the values seem to be consistently above the corresponding squareroot value.

Reasonable guess would be to assume that $\sqrt{n}$ is a lower limit for $\phi(n)$ when $n \rightarrow \infty$. With more precise examination, we see that is indeed the case.

\end{document}
