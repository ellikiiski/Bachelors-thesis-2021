\documentclass{article}
\usepackage[utf8]{inputenc}
\usepackage{parskip}
\usepackage{amsmath}
\usepackage{amssymb}

\title{Kandikaatopaikka}
\author{Elli Kiiski}
\date{\today}

\begin{document}
{\large
Elli Kiiski
\par
\textbf{2021 Kandikaatopaikka}
}
\vspace{0.5cm}

\section{Hardy-Wrightin todistuksen perkaamista}

\textit{G. H. Hardyn} ja \textit{E. M. Wrightin} kirjan \textit{An Introduction to the theory of numbers} sivulla 469 olevan $\phi$-funktion alarajan todistuksen läpikäyntiä.

\subsection{Määrittely: mitä todistetaan}

Aloitetaan määrittelemällä kuvaus
\begin{equation*}
    f(n)= \frac{\phi(n)e^\gamma \log\log n}{n},
\end{equation*}
missä $\gamma$ on Eulerin vakio.

% Mikä on Eulerin vakio ja mistä se hyppää tähän
% Mistä myöskään e yhtäkkiä hyppää tähän

Halutaan todistaa, että $\liminf f(n)=1$, mikä on yhtäpitävää sen kanssa, että $\phi$-funktion alaraja on $\frac{n}{e^\gamma \log\log n}$.

\subsection{Määrittely: miten todistetaan}

Riittää löytää funktiot $F_1(t)$ ja $F_2(t)$, joille pätee
\begin{enumerate}
\label{ehdot}
    \item $\lim_{t\rightarrow \infty} F_1(t) = 1$ ja $\lim_{t\rightarrow \infty} F_2(t) = 1$
    \item $f(n) \geq F_1(\log n)$ kaikilla $n\geq 3$
    \item $f(n_j) \leq \frac{1}{F_2(j)}$ äärettömällä kasvavalla jonolla $n_2, n_3,...$
\end{enumerate}

% Miksi tämä riittää??
% Liittynee majorantti- ja minoranttiperiaatteisiin tms.

Kirjan todistuksessa todistetaan myös sigmafunktion $\sigma(n)$ yläraja ja käytetään sitä hyväksi funktion $F_2$ löytämiseksi. Kysymys kuuluu, että tarvitaanko sigmafunktio välttämättä, että saadaan $\phi$-funktion alaraja todistetuksi. Se on kuitenkin vasta todistuksen toisen osan ongelma, joten jätetään se hetkeksi sikseen.

\subsection{Itse todistus osa 1: $f(n) \geq F_1(\log n)$}

Olkoot $p_1,p_2,...,p_{r-\rho} \leq \log n$ ja $p_{r-\rho+1},...,p_r > \log n$ luvun $n$ alkutekijöitä. Siis luvulla $n$ on yhteensä $r$ alkutekijää, joista $\log n$:ää suurempia on $\rho$ kappaletta.

Nyt
\begin{equation}
    (\log n)^\rho < p_{r-\rho+1} \cdot p_{r-\rho+2} \cdot \cdot \cdot p_r \leq n,
\end{equation}
mistä seuraa
\begin{equation}
    \rho < \frac{\log n}{\log\log n}.
\end{equation}

Eli $\log n$:ää suurempia alkulukutekijöitä on alle $\frac{\log n}{\log\log n}$ kappaletta.
Nyt tulokaavaa käyttäen $\phi$-funktion suhde $n$:ään voidaan ilmaista seuraavasti
\begin{align}
    \frac{\phi(n)}{n} & = \prod_{i=1}^r(1-\frac{1}{p_i})\\
    & = \prod_{i=1}^{r-\rho}(1-\frac{1}{p_i}) \prod_{i=r-\rho+1}^r(1-\frac{1}{p_i})\\
    & \geq \left(\prod_{i=1}^{r-\rho}(1-\frac{1}{p_i})\right) (1-\frac{1}{\log n})^\rho\\
    & > \left(\prod_{i=1}^{r-\rho}(1-\frac{1}{p_i})\right) (1-\frac{1}{\log n})^\frac{\log n}{\log \log n}.
\end{align}

Näin ollen voidaan valita
\begin{equation*}
    F_1(t)=e^\gamma \log t \left(1-\frac{1}{t}\right)^\frac{t}{\log t} \prod_{p\leq t} \left(1-\frac{1}{p}\right),
\end{equation*}
jolloin
\begin{align*}
    F_1(\log n) & = e^\gamma \log \log n \left(1-\frac{1}{\log n}\right)^\frac{\log n}{\log \log n} \prod_{p\leq \log n} \left(1-\frac{1}{p}\right)\\
    & = e^\gamma \log \log n \left(1-\frac{1}{\log n}\right)^\frac{\log n}{\log \log n} \prod_{i=1}^{r-\rho} \left(1-\frac{1}{p}\right)\\
    & \leq \frac{\phi(n)}{n} e^\gamma \log\log n = f(n).
\end{align*}

Kuitenkin funktiolle $F_1$ pätee
\begin{equation*}
    \lim_{t \rightarrow \infty} F_1(t) = \lim_{t \rightarrow \infty} e^\gamma \log t \left(1-\frac{1}{t}\right)^\frac{t}{\log t} \prod_{p\leq t} \left(1-\frac{1}{p}\right) = 1
\end{equation*}
(tähän tarvitaan todistus!! kirjassa viitattu todistukseen nro. 429),

Täten ensimmäinen osa on toidstettu.

\subsection{Itse todistus osa 2: $f(n_j) \leq \frac{1}{F_2(j)}$}

Tämä ei lienekään ihan niin iisi keissi.

\end{document}
